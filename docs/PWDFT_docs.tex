\documentclass[a4paper,10pt]{paper}

\usepackage[a4paper]{geometry}
%\geometry{verbose,tmargin=1.5cm,bmargin=1.5cm,lmargin=1.5cm,rmargin=1.5cm}

\setlength{\parskip}{\smallskipamount}
\setlength{\parindent}{0pt}

\usepackage{hyperref}
\usepackage{url}
\usepackage{xcolor}

\usepackage{amsmath}
\usepackage{amssymb}
\usepackage{braket}

\usepackage{minted}
\newminted{julia}{breaklines}
\newminted{text}{breaklines}

\newcommand{\jlcode}[1]{\mintinline{julia}{#1}}

\newenvironment{markdown}%
    {\VerbatimEnvironment\begin{VerbatimOut}{tmp.markdown}}%
    {\end{VerbatimOut}%
        \immediate\write18{pandoc tmp.markdown -t latex -o tmp.tex}%
        All atomic systems are assumed to be periodic.

The definition of type \texttt{Atoms} is given below.

\begin{Shaded}
\begin{Highlighting}[]
\NormalTok{mutable struct Atoms}
\NormalTok{    Natoms::}\DataTypeTok{Int64}
\NormalTok{    Nspecies::}\DataTypeTok{Int64}
\NormalTok{    positions::}\DataTypeTok{Array}\NormalTok{\{}\DataTypeTok{Float64}\NormalTok{,}\FloatTok{2}\NormalTok{\}}
\NormalTok{    atm2species::}\DataTypeTok{Array}\NormalTok{\{}\DataTypeTok{Int64}\NormalTok{,}\FloatTok{1}\NormalTok{\}}
\NormalTok{    atsymbs::}\DataTypeTok{Array}\NormalTok{\{}\DataTypeTok{String}\NormalTok{,}\FloatTok{1}\NormalTok{\}}
\NormalTok{    SpeciesSymbols::}\DataTypeTok{Array}\NormalTok{\{}\DataTypeTok{String}\NormalTok{,}\FloatTok{1}\NormalTok{\}}
\NormalTok{    LatVecs::}\DataTypeTok{Array}\NormalTok{\{}\DataTypeTok{Float64}\NormalTok{,}\FloatTok{2}\NormalTok{\}}
\NormalTok{    Zvals::}\DataTypeTok{Array}\NormalTok{\{}\DataTypeTok{Float64}\NormalTok{,}\FloatTok{1}\NormalTok{\}}
\KeywordTok{end}
\end{Highlighting}
\end{Shaded}

Information about \texttt{LatVecs} and \texttt{Zvals} are also available
from \texttt{PWGrid} and \texttt{PsPots}. They are included to reduce
number of required arguments to several functions.

Currently, the following functions are provided to initialize an
\texttt{Atoms}:

\begin{Shaded}
\begin{Highlighting}[]
\NormalTok{atoms = Atoms() }\CommentTok{# dummy constructor}
\NormalTok{atoms = init_atoms_xyz(filexyz; in_bohr=false, verbose=false)}
\NormalTok{atoms = init_atoms_xyz_string(filexyz; in_bohr=false, verbose=false)}
\end{Highlighting}
\end{Shaded}

Note that, \texttt{LatVecs} must be set manually by:

\begin{Shaded}
\begin{Highlighting}[]
\NormalTok{atoms.LatVecs = }\FloatTok{16}\NormalTok{*eye(}\FloatTok{3}\NormalTok{) }\CommentTok{# for example}
\end{Highlighting}
\end{Shaded}

\texttt{Zvals} is set when constructing \texttt{PWHamiltonian}.

Equation \begin{equation}
\frac{\alpha}{\beta}
\end{equation}
}

\begin{document}

\title{\textsf{PWDFT.jl} Documentation}
\author{Fadjar Fathurrahman}
\maketitle

\tableofcontents

\part{User Guide}

\section{Overview}

\textsf{PWDFT.jl} is a Julia package to carry out (electronic)
density functional theory calculations on materials.
It uses plane wave basis sets to discretize Kohn-Sham equations.
It also uses pseudopotentials to replace strong Coulombic nuclei-electron
interactions \footnote{You can carry out the calculation with
full potential, however this will requires a lot of plane waves (large
cutoff energy)}.

I assumes that the readers already have some experience with similar electronic
structure programs such as Quantum Espresso, ABINIT, VASP, and others.

\textsf{PWDFT.jl} uses Hartree atomic units internally.

\section{Installation}

You need the following to use \texttt{PWDFT.jl}:
\begin{itemize}
\item Julia version 0.7 or higher
\item LibXC
\item SPGLIB
\item C compiler to compiler LibXC and SPGLIB
\end{itemize}

\textsf{PWDFT.jl} is a Julia package, so you need to copy (or clone)
under Julia's package depot so that it can be used within Julia.
Presently, it is not yet a registered package. There are at least
two ways of doing this.

The first way is using Julia's package manager. The following command
can be issued at the Julia's console.
\begin{juliacode}
Pkg.add(PackageSpec(url="https://github.com/f-fathurrahman/PWDFT.jl"))
\end{juliacode}

The second way is to use Julia's development directory located at
\texttt{\$HOME/.julia/dev}.
I usually do this by sym-linking my original \textsf{PWDFT.jl} directory
to \texttt{\$HOME/.julia/dev}.


\section{Status}

\textbf{28 May 2018} The following features are working now:
\begin{itemize}
\item LDA and GGA, spin-paired and spin polarized calculations
\item Calculation with k-points (for periodic solids).
  \textsf{SPGLIB} is used to reduce the Monkhorst-Pack grid points
  for integration over Brillouin zone.
\end{itemize}

Band structure calculation is possible in principle as this can be
done by simply solving
Schrodinger equation with converged Kohn-Sham potentials, however there
is currently no tidy script or function to do that.

Total energy result for isolated systems (atoms and molecules) agrees quite
well with ABINIT and PWSCF results.

Total energy result for periodic solid is quite different from ABINIT and PWSCF.
I suspect that this is related to treatment of electrostatic terms in periodic system.

SCF is rather shaky for several systems, however it is working in quite well in nonmetallic
system.

\part{Implementation Notes}

In this note, I will describe several user-defined types that are
used throughout \textsf{PWDFT.jl}. Beginners are not expected to know them
all, however they are needed to know the internal of \textsf{PWDFT.jl}.

When referring to a file, I always meant to the file contained in
the \texttt{src/} directory.

Even though it is not strictly required in Julia, I tried to always use type annotations.

\section{Atoms}

\texttt{Atoms} can be used to represent molecular or crystalline structures.
This type is implemented in the file \texttt{src/Atoms.jl}.

It has the following fields:
\begin{itemize}
%
\item \jlcode{Natoms::Int64}: number of atoms present in the system
%
\item \jlcode{Nspecies::Int64}: number of atomic species present in the system
%
\item \jlcode{positions::Array{Float64,2}}: An array containing coordinates of atoms
  in bohr units. Its shape is \jlcode{(3,Natoms)}
%
\item \jlcode{atm2species::Array{Int64,1}}: An array containing mapping between atom index
  to species index. Its shape is \jlcode{(Natoms,)}
%
\item \jlcode{atsymbs::Array{String,1}}: An array containing atomic symbols for each atoms.
  Its shape is \jlcode{(Natoms,)}.
%
\item \jlcode{SpeciesSymbols::Array{String,1}}: An array containing unique symbols
  for each atomic species present in the system. Its shape is \jlcode{(Nspecies,)}.
%
\item \jlcode{LatVecs::Array{Float64,2}}: A 3 by 3 matrix describing lattice vectors for unit
  cell of the system.
%
\item \jlcode{Zvals::Array{Float64,1}}: An array containing number of (valence)
  electrons of each atomic species. Its shape is \verb|(Nspecies,)|.
\end{itemize}

An instance of \jlcode{Atoms} can be initialized using any of the following ways:

\begin{itemize}
\item Using \jlcode{Atoms} constructor, which has the following signature
%
\begin{juliacode}
function Atoms( ;xyz_file="", xyz_string="", xyz_string_frac="",
                in_bohr=false,
                LatVecs=10*[1.0 0.0 0.0; 0.0 1.0 0.0; 0.0 0.0 1.0] )
\end{juliacode}
%
\item The above mentioned \jlcode{Atoms} constructor is actually a wrapper for the functions
  \jlcode{init_atoms_xyz} and \jlcode{init_atoms_xyz_string} which takes either
  a string containing path to xyz file or the content of the xyz file
  itself. When using this function, one must set \jlcode{LatVecs}.
  field manually.
\end{itemize}

A note about \jlcode{Zvals}: Both \jlcode{Atoms} and \jlcode{init_atoms_*} function set
\jlcode{Zvals} to \jlcode{zeros(Nspecies)}. After passed to \jlcode{Hamiltonian} constructor
\jlcode{Zvals} will be set according to the pseudopotentials used.

TODO: examples

\section{Hamiltonian}

An instance \jlcode{Hamiltonian} is a central object in \textsf{PWDFT.jl}.
It is used to store various instances of other important types
such as atoms, plane wave grids, pseudopotentials, etc.
It is implemented in the file \texttt{src/Hamiltonian.jl}.

To create an instance of \jlcode{Hamiltonian}, we normaly need to provide at least
three arguments to the \jlcode{Hamiltonian} constructor:
%
\begin{itemize}
%
\item \jlcode{atoms::Atoms}: an instance of \jlcode{Atoms}
%
\item \jlcode{pspfiles::Array{String,1}}: a list of strings specifying the
  locations of pseudopotentials used in the
  calculations. Note that, the order should be the same as species ordering
  of \jlcode{atoms}, i.e. \jlcode{pspfiles[isp]} is the path of
  pseudopotentials of species with symbols \jlcode{atoms.SpeciesSymbols[isp]}.
%
\item \jlcode{ecutwfc::Float64}: cutoff energy for wave function
  expansion using plane wave basis set.
\end{itemize}

A simplified version of \jlcode{Hamiltonian} constructor only needs two arguments:
\jlcode{atoms::Atoms} and \jlcode{ecutwfc::Float64}. In this case, full Coulomb potential
will be used. We usually need very large cutoff energy in this case (probably
in the order or $10^2$ Hartree to obtain good convergence).

The structure of \jlcode{Hamiltonian} is designed such that we can perform
application or multiplication of Hamiltonian to wave function:
%
\begin{juliacode}
Hpsi = op_H(H, psi)
\end{juliacode}
%
or, (by overloading the \jlcode{*} operator
\footnote{The operator * is actually implemented as function in Julia})
%
\begin{juliacode}
Hpsi = H*psi
\end{juliacode}


\section{Plane wave basis}

The plane wave basis is described by the type \jlcode{PWGrid}.
This type is defined in the file \jlcode{PWGrid.jl}. It has the following fields:
\begin{itemize}
%
\item \jlcode{ecutwfc::Float64}: cutoff for wave function expansion
%
\item \jlcode{ecutrho::Float64}: cutoff for electron density expansion, for norm-converving
  pseudopotential: \jlcode{ecutrho = 4*ecutwfc}.
%
\item \jlcode{Ns::Tuple{Int64,Int64,Int64}}: parameters defining real-space grid points.
%
\item \jlcode{LatVecs::Array{Float64,2}}: lattice vectors of unit cell ($3\times3$ matrix)
%
\item \jlcode{RecVecs::Array{Float64,2}}: reciprocal lattice vectors ($3\times3$ matrix)
%
\item \jlcode{CellVolume::Float64}: the volume of real-space unit cell
%
\item \jlcode{r::Array{Float64,2}}: real-space grid points.
  Its shape is \jlcode{(3,Npoints)}
%
\item \jlcode{gvec::GVectors}: an instace of \jlcode{GVectors}:
  for potentials and density expansion
%
\item \jlcode{gvecw::GVectorsW}: an instace of \jlcode{GVectorsW}, for wave function expansion
%
\item \jlcode{planfw::FFTW.cFFTWPlan{Complex{Float64},-1,false,3}}: FFTW forward plan
%
\item \jlcode{planbw::AbstractFFTs.}\\
      \jlcode{ScaledPlan{ComplexF64,FFTW.cFFTWPlan{ComplexF64,1,false,3- Float64}}}: FFTW backward plan
%
\end{itemize}

The following constructor can be used to create an instance of `PWGrid`:

\begin{juliacode}
function PWGrid( ecutwfc::Float64,
                 LatVecs::Array{Float64,2};
                 kpoints=nothing )
\end{juliacode}


\subsection{$\mathbf{G}$-vectors for potentials and densities expansion}

$\mathbf{G}$-vectors are described by type `GVectors`. It is defined in file `PWGrid.jl`.
It has the following fields:
\begin{itemize}
%
\item \jlcode{Ng::Int64}: total number of $\mathbf{G}$-vectors
%
\item \jlcode{G::Array{Float64,2}} The array containing the actual $\mathbf{G}$-vectors.
   Its shape is \jlcode{(3,Ng)}.
%
\item \jlcode{G2::Array{Float64,1}}: The array containing magnitude of $\mathbf{G}$-vectors.
  Its shape is \jlcode{(Ng,)}.
%
\item \jlcode{idx_g2r::Array{Int64,1}} The array containing mapping between $\mathbf{G}$-vectors
  to real space grid points. Its shape is \jlcode{(Ng,)}.
%
\end{itemize}

The following function is used as the constructor:

\begin{juliacode}
function init_gvec( Ns, RecVecs, ecutrho )
\end{juliacode}
%
This function takes the following arguments
%
\begin{itemize}
%
\item \jlcode{Ns}: a tuple of three \jlcode{Int64}'s specifying sampling points
  along the 1st, 2nd, and 3rd lattice vector directions.
%
\item \jlcode{RecVecs}: 3 by 3 matrix describing reciprocal lattice vectors
%
\item \jlcode{ecutrho}: cutoff energy (in Hartree). For norm-conserving PP,
  it is 4 times `ecutwfc`.
\end{itemize}


\subsection{G-vectorsW}

The $\mathbf{G}$-vectors for wave function expansion is described by
the type \jlcode{GVectorsW}.
They are a subset of \jlcode{GVectors}. It has the following fields:
\begin{itemize}
%
\item \jlcode{Ngwx::Int64}: maximum number of $\mathbf{G}$-vectors for all kpoints.
%
\item \jlcode{Ngw::Array{Int64,1}}: number of $\mathbf{G}$-vectors for each kpoints
%
\item \jlcode{idx_gw2g::Array{Array{Int64,1},1}}: mapping between indices of
  \jlcode{GVectorsW} to indices of \jlcode{GVectors}.
%
\item \jlcode{idx_gw2r::Array{Array{Int64,1},1}}: mapping between indices of `GVectorsW`
  to indices of real space grid points \jlcode{PWGrid.r}
%
\item \jlcode{kpoints::KPoints}: an instance of \jlcode{KPoints}
%
\end{itemize}


Constructor: TODO

\subsection{Bloch wave vector}

The type describing Bloch wave vectors, or commonly referred
to as k-points, is \jlcode{KPoints}. It is defined in the file
\jlcode{KPoints.jl}. It has the following fields
%
\begin{itemize}
%
\item \jlcode{Nkpt::Int64}: total number of k-points.
%
\item \jlcode{k::Array{Float64,2}}: the actual k-points.
  Its shape is \jlcode{(3,Nkpt)}.
%
\item \jlcode{wk::Array{Float64,1}}: the weight of each k-points needed
  for integration over Brillouin zone
%
\item \jlcode{RecVecs::Array{Float64,2}}: a copy of \jlcode{PWGrid.RecVecs}.
\end{itemize}

Monkhorst-Pack grid points:
%
\begin{equation}
k_{i} = \frac{2n_i -N_i - 1}{2N_i} \mathbf{b}_i.\hspace{1em}
i = 1,2,3
\end{equation}
%
where $n_i=1,2,...,N_{i}$
size = \((N_1, N_2, N_3)\) and the \(\mathbf{b}_i\)'s are reciprocal lattice vectors.


Constructor:

\begin{juliacode}
function KPoints( atoms::Atoms, mesh::Array{Int64,1}, is_shift::Array{Int64,1};
                  time_reversal=1 )
\end{juliacode}

SPGLIB is used internally to generate reduce number of k-points.



\section{Wave functions}

Using `\jlcode{Array{ComplexF64,2}}.

General wavefunction on kpoints


\section{Total energy components}

Total energy components are stored in the type \jlcode{Energies}. Currently its
fields are as follows.
\begin{itemize}
%
\item `Kinetic`: kinetic energy
%
\item \jlcode{Ps_loc}: local pseudopotential energy
%
\item \jlcode{Ps_nloc}: nonlocal pseudopotential energy
%
\item \jlcode{Hartree}: classical electrostatic energy
%
\item \jlcode{XC}: exchange correlation energy
%
\item \jlcode{NN}: nuclear-nuclear (repulsive) interaction energy
%
\item \jlcode{PspCore}: core (screened) pseudopotential energy
%
\item \jlcode{mTS}: electronic entropy contribution (with minus sign). This is
  used for calculation with partial occupations such as metals.
%
\end{itemize}


\section{Local potentials}

Various local potentials are stored in the type \jlcode{Potentials}. Currently
its fields are as follows.
\begin{itemize}
%
\item \jlcode{Ps_loc} local pseudopotential components.
  Its shape is \jlcode{(Npoints,)}.
%
\item \jlcode{Hartree}: classical electrostatic potential.
  Its shape is \jlcode{(Npoints,)}.
%
\item \jlcode{XC}: exchange-correlation potential. This potential can be spin
  dependent so its shape is \jlcode{(Npoints,Nspin)}.
%
\end{itemize}

All of these potentials are in the real space representation.


\section{Pseudopotentials}

Currently, only GTH pseudopotentials with no core-correction are supported.
The type for handling GTH pseudopotential for a species is \jlcode{PsPot_GTH}.
It is declared as follows.

\begin{juliacode}
struct PsPot_GTH
    pspfile::String
    atsymb::String
    zval::Int64
    rlocal::Float64
    rc::Array{Float64,1}
    c::Array{Float64,1}
    h::Array{Float64,3}
    lmax::Int64
    Nproj_l::Array{Int64,1}
    rcut_NL::Array{Float64,1}
end
\end{juliacode}


\section{Solving Kohn-Sham problems}

Two main algorithms: SCF and direct minimization

Function names: \jlcode{KS_solve_*}

\subsection{Self-consistent field}

\jlcode{KS_solve_SCF}

\subsection{Eigensolver}

Davidson, LOBPCG, and PCG

\subsection{Direct energy minimization via nonlinear conjugate gradient method}

\jlcode{KS_solve_Emin_PCG}

This method is described by IsmailBeigi-Arias

\subsection{Direct constrained minimization method of Yang}

\jlcode{KS_solve_DCM}

Chebyshev filtered subspace iteration SCF

\jlcode{update_psi="CheFSI"}


\end{document}
