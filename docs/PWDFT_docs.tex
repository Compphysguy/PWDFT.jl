\documentclass[a4paper,10pt]{article}

\usepackage[a4paper]{geometry}
\geometry{verbose,tmargin=2.5cm,bmargin=2.5cm,lmargin=2.5cm,rmargin=2.5cm}

\setlength{\parskip}{\smallskipamount}
\setlength{\parindent}{0pt}

\usepackage{fontspec}
\setmonofont{FreeMono}
%\setmonofont{Source Code Pro}

\usepackage{hyperref}
\usepackage{url}
\usepackage{xcolor}

\usepackage{amsmath}
\usepackage{amssymb}
\usepackage{braket}

\usepackage[normalem]{ulem}

\usepackage{mhchem}

\usepackage{minted}
\newminted{julia}{breaklines}
\newminted{text}{breaklines}

\newcommand{\jlinline}[1]{\mintinline{julia}{#1}}
\newcommand{\txtinline}[1]{\mintinline{text}{#1}}

\newenvironment{markdown}%
    {\VerbatimEnvironment\begin{VerbatimOut}{tmp.markdown}}%
    {\end{VerbatimOut}%
        \immediate\write18{pandoc tmp.markdown -t latex -o tmp.tex}%
        All atomic systems are assumed to be periodic.

The definition of type \texttt{Atoms} is given below.

\begin{Shaded}
\begin{Highlighting}[]
\NormalTok{mutable struct Atoms}
\NormalTok{    Natoms::}\DataTypeTok{Int64}
\NormalTok{    Nspecies::}\DataTypeTok{Int64}
\NormalTok{    positions::}\DataTypeTok{Array}\NormalTok{\{}\DataTypeTok{Float64}\NormalTok{,}\FloatTok{2}\NormalTok{\}}
\NormalTok{    atm2species::}\DataTypeTok{Array}\NormalTok{\{}\DataTypeTok{Int64}\NormalTok{,}\FloatTok{1}\NormalTok{\}}
\NormalTok{    atsymbs::}\DataTypeTok{Array}\NormalTok{\{}\DataTypeTok{String}\NormalTok{,}\FloatTok{1}\NormalTok{\}}
\NormalTok{    SpeciesSymbols::}\DataTypeTok{Array}\NormalTok{\{}\DataTypeTok{String}\NormalTok{,}\FloatTok{1}\NormalTok{\}}
\NormalTok{    LatVecs::}\DataTypeTok{Array}\NormalTok{\{}\DataTypeTok{Float64}\NormalTok{,}\FloatTok{2}\NormalTok{\}}
\NormalTok{    Zvals::}\DataTypeTok{Array}\NormalTok{\{}\DataTypeTok{Float64}\NormalTok{,}\FloatTok{1}\NormalTok{\}}
\KeywordTok{end}
\end{Highlighting}
\end{Shaded}

Information about \texttt{LatVecs} and \texttt{Zvals} are also available
from \texttt{PWGrid} and \texttt{PsPots}. They are included to reduce
number of required arguments to several functions.

Currently, the following functions are provided to initialize an
\texttt{Atoms}:

\begin{Shaded}
\begin{Highlighting}[]
\NormalTok{atoms = Atoms() }\CommentTok{# dummy constructor}
\NormalTok{atoms = init_atoms_xyz(filexyz; in_bohr=false, verbose=false)}
\NormalTok{atoms = init_atoms_xyz_string(filexyz; in_bohr=false, verbose=false)}
\end{Highlighting}
\end{Shaded}

Note that, \texttt{LatVecs} must be set manually by:

\begin{Shaded}
\begin{Highlighting}[]
\NormalTok{atoms.LatVecs = }\FloatTok{16}\NormalTok{*eye(}\FloatTok{3}\NormalTok{) }\CommentTok{# for example}
\end{Highlighting}
\end{Shaded}

\texttt{Zvals} is set when constructing \texttt{PWHamiltonian}.

Equation \begin{equation}
\frac{\alpha}{\beta}
\end{equation}
}

\definecolor{mintedbg}{rgb}{0.90,0.90,0.90}
\usepackage{mdframed}
\BeforeBeginEnvironment{minted}{\begin{mdframed}[backgroundcolor=mintedbg,%
  rightline=false,leftline=false,topline=false,bottomline=false]}
\AfterEndEnvironment{minted}{\end{mdframed}}

\begin{document}

\title{\textsf{PWDFT.jl} Documentation}
\author{Fadjar Fathurrahman}
\date{}
\maketitle

%\tableofcontents

\textbf{This document is a work in progress}

In this part I will describe my design choices in implementing \textsf{PWDFT.jl}.
This design is by no means perfect
and it might change in the future to accomodate more complex use cases.

\section{Overview}

The design of \textsf{PWDFT.jl} is intended to be rather simple. One constraint
that is set to the code is that it should be possible to perform application
of Hamiltonian operator to wave function as simple as:
%
\begin{juliacode}
Hpsi = Ham*psi # or
Hpsi = op_H(Ham, psi)
\end{juliacode}
%
where \jlinline{psi} is, currently, of type \jlinline{Array{ComplexF64,2}}
\footnote{This function may be extended take other types other that plain Julia
array for more complex case.}.
%
This comes with an important consequences: all other pieces of information
about how this operation is done should be present in the type of \jlinline{Ham}.
\footnote{We will also see some quirks related to this design choice later,
such as applying Hamiltonian to several k-points or spin-polarized case}.

In \textsf{PWDFT.jl}, the type of \jlinline{Ham} is \jlinline{Hamiltonian}.
Several important fields of \jlinline{Hamiltonian} are instances of the following
types (please refer to the source code for more details about this):
\begin{itemize}
\item \jlinline{Atoms}: contains information about atomic structure: cell
vectors, atomic species and atomic coordinates.
\item \jlinline{PsPot_GTH}: contains information about atomic pseudopotentials.
\item \jlinline{Electrons}: contains information about electronic states.
\item \jlinline{PWGrid}: contains information about plane wave basis set.
\item \jlinline{Potentials}: contains information about local potentials such
as local pseudopotential, Hartree and exchange-correlation potential.
\item \jlinline{PsPotNL}: contains information about nonlocal pseudopotential
terms.
\item \jlinline{Energies}: contains information about components of Kohn-Sham
energy.
\item \jlinline{SymmetryInfo}: contains information about symmetry operations.
\end{itemize}


\section{Describing atomic structure}
%
The type \jlinline{Atoms} contains the following information:
%
\begin{itemize}
\item Number of atoms: \jlinline{Natoms::Int64}
\item Number of atomic species: \jlinline{Nspecies::Int64}
\item Atomic coordinates: \jlinline{positions::Array{Float64,2}}
\item Unit cell vectors (lattice vectors): \jlinline{LatVecs::Array{Float64,2}}
\end{itemize}
%
\jlinline{Atoms} also contains several other fields such as \jlinline{Zvals}
which will be set according to the pseudopotentials assigned to
the instance of \jlinline{Atoms}.
\footnote{Maybe we should include pseudopotential information under the
\jlinline{Atoms} type. However this would make \jlinline{Atoms} "heavier".}


\jlinline{LatVecs} is a $3\times3$ matrix. The vectors are stored column-wise which is
opposite to the PWSCF input convention.
Convenience functions to calculate lattice vectors for several types of Bravais lattice
are provided in \textsf{PWDFT.jl}. These functions adapt PWSCF definition. Several
of these functions are listed below:
\begin{itemize}
\item \jlinline{gen_lattice_sc} or \jlinline{gen_lattice_cubic} for generating
simple cubic lattice vectors.
\item \jlinline{gen_lattice_fcc}: for fcc structure
\item \jlinline{gen_lattice_bcc}: for bcc structure
\item \jlinline{gen_lattice_hcp}: for hcp structure
\end{itemize}
Please see file \txtinline{gen_lattice.jl} for more information.


There are several ways to initialize an instance of \jlinline{Atoms}. The following
are typical cases.
%
\begin{itemize}
%
\item From xyz file. We need to supply the path to xyz file as string and
set the lattice vectors:
%
\begin{juliacode}
atoms = Atoms(xyz_file="file.xyz", LatVecs=gen_lattice_sc(16.0))
\end{juliacode}
%
\item For crystalline systems, using keyword argument \jlinline{xyz_string_frac}
is sometimes convenient:
%
\begin{juliacode}
atoms = Atoms(xyz_string_frac=
        """
        2

        Si  0.0  0.0  0.0
        Si  0.25  0.25  0.25
        """, in_bohr=true,
        LatVecs=gen_lattice_fcc(10.2631))
\end{juliacode}
%
\textbf{IMPORTANT} We need to be careful to also specify \jlinline{in_bohr} keyword to get
the correct coordinates in bohr (which is used internally in \jlinline{PWDFT.jl}).
%
\item From extended xyz file, the lattice vectors information is included
along with several others information, if any, however they are ignored):
%
\begin{juliacode}
atoms = Atoms(ext_xyz_file="file.xyz")
\end{juliacode}
%
\end{itemize}


\section{Describing plane wave basis set and real space grid}

The type \jlinline{PWGrid} wraps various variables related to plane wave basis
set.

Real space grid points:
$$
\mathbf{r} = \frac{i}{N_{s1}}\mathbf{a}_{1} + \frac{j}{N_{s2}}\mathbf{a}_{2} +
\frac{k}{N_{s3}}\mathbf{a}_{3}
$$

$i = 0,1,\ldots,N_{s1}-1$

$j = 0,1,\ldots,N_{s2}-1$

$k = 0,1,\ldots,N_{s3}-1$


\section{Pseudopotentials}
%
Currently, \textsf{PWDFT.jl} supports a subset of GTH (Goedecker-Teter-Hutter)
pseudopotentials. This type of pseudopotential is analytic and thus is somewhat
easier to program.
%
\textsf{PWDFT.jl} distribution contains several parameters
of GTH pseudopotentials for LDA and GGA functionals.

These pseudopotentials can be written in terms of
local $V^{\mathrm{PS}}_{\mathrm{loc}}$ and
angular momentum $l$ dependent
nonlocal components $\Delta V^{\mathrm{PS}}_{l}$:
\begin{equation}
V_{\mathrm{ene-nuc}}(\mathbf{r}) =
\sum_{I} \left[
V^{\mathrm{PS}}_{\mathrm{loc}}(\mathbf{r}-\mathbf{R}_{I}) +
\sum_{l=0}^{l_{\mathrm{max}}}
V^{\mathrm{PS}}_{l}(\mathbf{r}-\mathrm{R}_{I},\mathbf{r}'-\mathbf{R}_{I})
\right]
\end{equation}

\subsection{Local pseudopotential}

The local pseudopotential for
$I$-th atom, $V^{\mathrm{PS}}_{\mathrm{loc}}(\mathbf{r}-\mathbf{R}_{I})$,
is radially symmetric
function with the following radial form
\begin{equation}
V^{\mathrm{PS}}_{\mathrm{loc}}(r) =
-\frac{Z_{\mathrm{val}}}{r}\mathrm{erf}\left[
\frac{\bar{r}}{\sqrt{2}} \right] +
\exp\left[-\frac{1}{2}\bar{r}^2\right]\left(
C_{1} + C_{2}\bar{r}^2 + C_{3}\bar{r}^4 + C_{4}\bar{r}^6
\right)
\label{eq:V_ps_loc_R}
\end{equation}
with $\bar{r}=r/r_{\mathrm{loc}}$ and $r_{\mathrm{loc}}$, $Z_{\mathrm{val}}$,
$C_{1}$, $C_{2}$, $C_{3}$ and $C_{4}$ are the corresponding pseudopotential
parameters.
In $\mathbf{G}$-space, the GTH local pseudopotential can be written as
\begin{multline}
V^{\mathrm{PS}}_{\mathrm{loc}}(G) = -\frac{4\pi}{\Omega}\frac{Z_{\mathrm{val}}}{G^2}
\exp\left[-\frac{x^2}{2}\right] +
\sqrt{8\pi^3} \frac{r^{3}_{\mathrm{loc}}}{\Omega}\exp\left[-\frac{x^2}{2}\right]\times\\
\left( C_{1} + C_{2}(3 - x^2) + C_{3}(15 - 10x^2 + x^4) + C_{4}(105 - 105x^2 + 21x^4 - x^6) \right)
\label{eq:V_ps_loc_G}
\end{multline}
where $x=G r_{\mathrm{loc}}$.

\subsection{Nonlocal pseudopotential}

The nonlocal component of GTH pseudopotential can written in real space as
\begin{equation}
V^{\mathrm{PS}}_{l}(\mathbf{r}-\mathbf{R}_{I},\mathbf{r}'-\mathbf{R}_{I}) =
\sum_{\mu=1}^{N_{l}} \sum_{\nu=1}^{N_{l}} \sum_{m=-l}^{l}
\beta_{\mu lm}(\mathbf{r}-\mathbf{R}_{I})\,
h^{l}_{\mu\nu}\,
\beta^{*}_{\nu lm}(\mathbf{r}'-\mathbf{R}_{I})
\end{equation}
where $\beta_{\mu lm}(\mathbf{r})$ are atomic-centered projector functions
\begin{equation}
\beta_{\mu lm}(\mathbf{r}) =
p^{l}_{\mu}(r) Y_{lm}(\hat{\mathbf{r}})
\label{eq:proj_NL_R}
\end{equation}
%
and $h^{l}_{\mu\nu}$ are the pseudopotential parameters and
$Y_{lm}$ are the spherical harmonics. Number of projectors per angular
momentum $N_{l}$ may take value up to 3 projectors.
%
In $\mathbf{G}$-space, the nonlocal part of GTH pseudopotential can be described by
the following equation.
\begin{equation}
V^{\mathrm{PS}}_{l}(\mathbf{G},\mathbf{G}') =
(-1)^{l} \sum_{\mu}^{3} \sum_{\nu}^{3}\sum_{m=-l}^{l}
\beta_{\mu l m}(\mathbf{G}) h^{l}_{\mu\nu}
\beta^{*}_{\nu l m}(\mathbf{G}')
\end{equation}
with the projector functions
\begin{equation}
\beta_{\mu lm}(\mathbf{G}) = p^{l}_{\mu}(G) Y_{lm}(\hat{\mathbf{G}})
\label{eq:betaNL_G}
\end{equation}
The radial part of projector functions take the following form
\begin{equation}
p^{l}_{\mu}(G) = q^{l}_{\mu}\left(Gr_{l}\right)
\frac{\pi^{5/4}G^{l}\sqrt{ r_{l}^{2l+3}}}{\sqrt{\Omega}}
\exp\left[-\frac{1}{2}G^{2}r^{2}_{l}\right]
\label{eq:proj_NL_G}
\end{equation}
%
For $l=0$, we consider up to $N_{l}=3$ projectors:
\begin{align}
q^{0}_{1}(x) & = 4\sqrt{2} \\
q^{0}_{2}(x) & = 8\sqrt{\frac{2}{15}}(3 - x^2) \\
q^{0}_{3}(x) & = \frac{16}{3}\sqrt{\frac{2}{105}} (15 - 20x^2 + 4x^4)
\end{align}
%
For $l=1$, we consider up to $N_{l}=3$ projectors:
\begin{align}
q^{1}_{1}(x) & = 8 \sqrt{\frac{1}{3}} \\
q^{1}_{2}(x) & = 16 \sqrt{\frac{1}{105}} (5 - x^2) \\
q^{1}_{3}(x) & = 8 \sqrt{\frac{1}{1155}} (35 - 28x^2 + 4x^4)
\end{align}
%
For $l=2$, we consider up to $N_{l}=2$ projectors:
\begin{align}
q^{2}_{1}(x) & = 8\sqrt{\frac{2}{15}} \\
q^{2}_{2}(x) & = \frac{16}{3} \sqrt{\frac{2}{105}}(7 - x^2)
\end{align}
%
For $l=3$, we only consider up to $N_{l}=1$ projector:
\begin{equation}
q^{3}_{1}(x) = 16\sqrt{\frac{1}{105}}
\end{equation}

In the present implementation, we construct the local and nonlocal
components of pseudopotential in the $\mathbf{G}$-space using
their Fourier-transformed expressions
and transformed them back to real space if needed.
We refer the readers to the original
reference \cite{Goedecker1996} and the book \cite{Marx2009}
for more information about GTH pseudopotentials.

Due to the separation of local and non-local components of electrons-nuclei
interaction, Equation \eqref{eq:E_ele_nuc} can be written as
\begin{equation}
E_{\mathrm{ele-nuc}} = E^{\mathrm{PS}}_{\mathrm{loc}}
+ E^{\mathrm{PS}}_{\mathrm{nloc}}
\end{equation}
%
The local pseudopotential contribution is
\begin{equation}
E^{\mathrm{PS}}_{\mathrm{loc}} =
\int_{\Omega} \rho(\mathbf{r})\,V^{\mathrm{PS}}_{\mathrm{loc}}(\mathbf{r})\,
\mathrm{d}\mathbf{r}
\end{equation}
%
and the non-local contribution is
\begin{equation}
E^{\mathrm{PS}}_{\mathrm{nloc}} =
\sum_{\mathbf{k}}
\sum_{i}
w_{\mathbf{k}}
f_{i\mathbf{k}}
\int_{\Omega}\,
\psi^{*}_{i\mathbf{k}}(\mathbf{r})
\left[
\sum_{I}\sum_{l=0}^{l_{\mathrm{max}}}
V^{\mathrm{PS}}_{l}(\mathbf{r}-\mathbf{R}_{I},\mathbf{r}'-\mathbf{R}_{I})
\right]
\psi_{i\mathbf{k}}(\mathbf{r})
\,\mathrm{d}\mathbf{r}.
\end{equation}


\appendix
\section{Howtos}

This part contains miscellaneous info.

\subsection*{Initializing \jlcode{Atoms}}

From xyz file: supply the path to xyz file as string and
set the lattice vectors:

\begin{juliacode}
atoms = Atoms(xyz_file="file.xyz", LatVecs=gen_lattice_sc(16.0))
\end{juliacode}

In extended xyz file, the lattice vectors information is included
(along with several others information, if any):

\begin{juliacode}
atoms = Atoms(ext_xyz_file="file.xyz")
\end{juliacode}

For crystalline systems, using keyword argument \jlcode{xyz_string_frac}
is sometimes convenient:

\begin{juliacode}
atoms = Atoms(xyz_string_frac=
        """
        2

        Si  0.0  0.0  0.0
        Si  0.25  0.25  0.25
        """, in_bohr=true,
        LatVecs=gen_lattice_fcc(10.2631))
\end{juliacode}

\textbf{IMPORTANT} We need to be careful to also specify \jlcode{in_bohr} keyword to get
the correct coordinates in bohr (which is used internally in \jlcode{PWDFT.jl}).



\subsection*{Referring or including files in \jlcode{sandbox} (or other dirs in \jlcode{PWDFT.jl})}

\begin{juliacode}
using PWDFT
const DIR_PWDFT = joinpath(dirname(pathof(PWDFT)),"..")
const DIR_PSP = joinpath(DIR_PWDFT,"pseudopotentials","pade_gth")
const DIR_STRUCTURES = joinpath(DIR_PWDFT, "structures")

pspfiles = [joinpath(DIR_PSP,"Ag-q11.gth")]
\end{juliacode}


\subsection*{Generating lattice vectors}

Lattice vectors are simply 3x3 array. We can set it manually or use
one of functions defined in \jlcode{gen_lattice_pwscf.jl}.
for generating lattice vectors for Bravais lattices that used
in Quantum ESPRESSO's PWSCF.

\subsection*{Using Babel to generate xyz file from SMILES}

\begin{textcode}
babel file.smi file.sdf
babel file.sdf file.xyz
\end{textcode}

Use \myverbcode{babel -h} to autogenerate hydrogens.



\subsection*{Setting up pseudopotentials}

One can use the function \jlcode{get_default_psp(::Atoms)} to get default
pseudopotentials set for a given instance of \jlcode{Atoms}.

Currently, it is not part of main \jlcode{PWDFT.jl} package. It is located
under \myverbcode{sandbox} subdirectory of \jlcode{PWDFT.jl} distribution.

\begin{juliacode}
using PWDFT

DIR_PWDFT = jointpath(dirname(pathof(PWDFT)),"..")
include(jointpath(DIRPWDFT,"sandbox","get_default_psp.jl"))

atoms = Atoms(ext_xyz_file="atoms.xyz")
pspfiles = get_default_psp(atoms)
\end{juliacode}

Alternatively, one can set \myverbcode{pspfiles} manually because it is simply
an array of \jlcode{String}:
\begin{juliacode}
pspfiles = ["Al-q3.gth", "O-q6.gth"]
\end{juliacode}

\textbf{IMPORTANT} Be careful to set the order of species to be same as
\jlcode{atoms.SpeciesSymbols}. For example, if
\begin{juliacode}
atoms.SpeciesSymbols = ["Al", "O", "H"]
\end{juliacode}
then
\begin{juliacode}
pspfiles = ["Al-q3.gth", "O-q6.gth", "H-q1.gth"]
\end{juliacode}

\subsection*{Initializing Hamiltonian}

For molecular systems:
\begin{juliacode}
Ham = Hamiltonian( atoms, pspfiles, ecutwfc )
\end{juliacode}

For insulator and semiconductor solids:
\begin{juliacode}
Ham = Hamiltonian( atoms, pspfiles, ecutwfc, meshk=[3,3,3] )
\end{juliacode}

For metallic systems:
\begin{juliacode}
Ham = Hamiltonian( atoms, pspfiles, ecutwfc, meshk=[3,3,3], extra_states=4 )
\end{juliacode}

Empty extra states can be specified by using \jlcode{extra_states} keyword.

For spin-polarized systems, \jlcode{Nspin} keyword can be used.


\subsection*{Iterative diagonalization of Hamiltonian}

\begin{juliacode}
evals =  diag_LOBPCG!( Ham, psiks, verbose=false, verbose_last=false,
                       Nstates_conv=Nstates_occ )
\end{juliacode}


\subsection*{Calculating electron density}

Several ways:
\begin{juliacode}
Rhoe = calc_rhoe( Nelectrons, pw, Focc, psiks, Nspin )
# or
Rhoe = calc_rhoe( Ham, psiks )
# or
calc_rhoe!( Ham, psiks, Rhoe )
\end{juliacode}

\subsection*{Read and write array (binary file)}

Write to binary files:
\begin{juliacode}
for ikspin = 1:Nkpt*Nspin
    wfc_file = open("WFC_ikspin_"*string(ikspin)*".data","w")
    write( wfc_file, psiks[ikspin] )
    close( wfc_file )
end
\end{juliacode}

Read from binary files:
\begin{juliacode}
psiks = BlochWavefunc(undef,Nkpt)
for ispin = 1:Nspin
for ik = 1:Nkpt
    ikspin = ik + (ispin-1)*Nkpt
    # Don't forget to use read mode
    wfc_file = open("WFC_ikspin_"*string(ikspin)*".data","r")
    psiks[ikspin] = Array{ComplexF64}(undef,Ngw[ik],Nstates)
    psiks[ikspin] = read!( wfc_file, psiks[ikspin] )
    close( wfc_file )
end
end
\end{juliacode}




\subsection*{Subspace rotation}

In case need sorting:
\begin{juliacode}
Hr = psiks[ikspin]' * op_H( Ham, psiks[ikspin] )
evals, evecs = eigen(Hr)
evals = real(evals[:])

# Sort in ascending order based on evals 
idx_sorted = sortperm(evals)

# Copy to Hamiltonian
Ham.electrons.ebands[:,ikspin] = evals[idx_sorted]

# and rotate
psiks[ikspin] = psiks[ikspin]*evecs[:,idx_sorted]
\end{juliacode}

Usually we don't need to sort the eigenvalues if we use Hermitian matrix. We can calculate the
subspace Hamiltonian by:
\begin{juliacode}
evals, evecs = eigen(Hermitian(Hr))
\end{juliacode}



\section*{Status}

\textbf{29 July 2019} Total energy results are now similar to ABINIT
and Quantum ESPRESSO. A rather comprehensive test has been added
for SCF and Emin PCG for several simple systems.


\textbf{28 May 2018} The following features are working now:
\begin{itemize}
\item LDA and GGA, spin-paired and spin polarized calculations
\item Calculation with k-points (for periodic solids).
  \textsf{SPGLIB} is used to reduce the Monkhorst-Pack grid points
  for integration over Brillouin zone.
\end{itemize}

Band structure calculation is possible in principle as this can be
done by simply solving
Schrodinger equation with converged Kohn-Sham potentials, however there
is currently no tidy script or function to do that.

Total energy result for isolated systems (atoms and molecules) agrees quite
well with ABINIT and PWSCF results.

\sout{Total energy result for periodic solid is quite different from ABINIT and PWSCF.
I suspect that this is related to treatment of electrostatic terms in periodic system.}

These discrepancies have been minimized. For several systems the agreement is very good
even though I did not use the same algorithm as ABINIT.

\sout{SCF is rather shaky for several systems, however it is working in quite well in nonmetallic
system.}

SCF stability has been improved with Pulay mixing and its variants.



\bibliographystyle{unsrt}
\bibliography{BIBLIO}



\end{document}
