\documentclass[a4paper,10pt]{article}

\usepackage[a4paper]{geometry}
\geometry{verbose,tmargin=2.5cm,bmargin=2.5cm,lmargin=2.5cm,rmargin=2.5cm}

\setlength{\parskip}{\smallskipamount}
\setlength{\parindent}{0pt}

\usepackage{amsmath}
\usepackage{amssymb}
\usepackage{braket}

\usepackage{fontspec}
\setmonofont{DejaVu Sans Mono}
%\setmonofont{Source Code Pro}

\usepackage{hyperref}
\usepackage{url}
\usepackage{xcolor}

\usepackage[normalem]{ulem}

\usepackage{mhchem}

\usepackage{minted}
\newminted{julia}{breaklines,fontsize=\footnotesize}
\newminted{text}{breaklines,fontsize=\footnotesize}


\newcommand{\txtinline}[1]{\mintinline[fontsize=\footnotesize]{text}{#1}}
\newcommand{\jlinline}[1]{\mintinline[fontsize=\footnotesize]{julia}{#1}}

\definecolor{mintedbg}{rgb}{0.90,0.90,0.90}
\usepackage{mdframed}

%\BeforeBeginEnvironment{minted}{\begin{mdframed}[backgroundcolor=mintedbg,%
%  rightline=false,leftline=false,topline=false,bottomline=false]}
%\AfterEndEnvironment{minted}{\end{mdframed}}


\usepackage{tikz}
\usetikzlibrary{shapes.geometric}
\tikzstyle{mybox} = [draw=blue, fill=green!5, very thick, rectangle,
  rounded corners, inner sep=10pt, inner ysep=20pt]
\tikzstyle{fancytitle} = [fill=blue, text=white]


\begin{document}

\title{\textsf{PWDFT.jl} Documentation}
\author{Fadjar Fathurrahman}
\date{}
\maketitle

%\tableofcontents

\textbf{This document is a work in progress}

In this part I will describe my design choices in implementing \textsf{PWDFT.jl}.
This design is by no means perfect
and it might change in the future to accomodate more complex use cases.

\section{Overview}

The design of \textsf{PWDFT.jl} is intended to be rather simple. One constraint
that is set to the code is that it should be possible to perform application
of Hamiltonian operator to wave function as simple as:
%
\begin{juliacode}
Hpsi = Ham*psi # or
Hpsi = op_H(Ham, psi)
\end{juliacode}
%
where \jlinline{psi} is, currently, of type \jlinline{Array{ComplexF64,2}}
\footnote{This function may be extended take other types other that plain Julia
array for more complex case.}.
%
This comes with an important consequences: all other pieces of information
about how this operation is done should be present in the type of \jlinline{Ham}.
\footnote{We will also see some quirks related to this design choice later,
such as applying Hamiltonian to several k-points or spin-polarized case}.

In \textsf{PWDFT.jl}, the type of \jlinline{Ham} is \jlinline{Hamiltonian}.
Several important fields of \jlinline{Hamiltonian} are instances of the following
types (please refer to the source code for more details about this):
\begin{itemize}
\item \jlinline{Atoms}: contains information about atomic structure: cell
vectors, atomic species and atomic coordinates.
\item \jlinline{PsPot_GTH}: contains information about atomic pseudopotentials.
\item \jlinline{Electrons}: contains information about electronic states.
\item \jlinline{PWGrid}: contains information about plane wave basis set.
\item \jlinline{Potentials}: contains information about local potentials such
as local pseudopotential, Hartree and exchange-correlation potential.
\item \jlinline{PsPotNL}: contains information about nonlocal pseudopotential
terms.
\item \jlinline{Energies}: contains information about components of Kohn-Sham
energy.
\item \jlinline{SymmetryInfo}: contains information about symmetry operations.
\end{itemize}


\section{Describing an atomic system}

\begin{markdown}
Because plane wave basis set is used, all atomic systems are assumed
to be periodic. For isolated molecular systems, a periodic bounding box
must be specified.

Currently, the definition of type `Atoms` is given below.

```julia
mutable struct Atoms
    Natoms::Int64
    Nspecies::Int64
    positions::Array{Float64,2}
    atm2species::Array{Int64,1}
    atsymbs::Array{String,1}
    SpeciesSymbols::Array{String,1}
    LatVecs::Array{Float64,2}
    Zvals::Array{Float64,1}
end
```

Information about `LatVecs` and `Zvals` are also available
from `PWGrid` and `PsPotGTH`, respectively.
They are included to reduce number of required arguments to several functions.

`LatVecs` represents lattice vectors $a_{1}$, $a_{2}$, dan $a_{3}$, stacked by column
in $3\times3$ matrix.

`Zvals` is set when constructing `PWHamiltonian`. The default value is `zeros(Nspecies)`.

Currently, the following functions are provided to initialize an `Atoms`:

- Using dummy constructor:
  ```julia
  atoms = Atoms()
  ```
  It is important to set other fields of `atoms` accordingly.

- Using xyz-like structure
  ```julia
  atoms = init_atoms_xyz(filexyz; in_bohr=false, verbose=false)
  atoms = init_atoms_xyz_string(filexyz; in_bohr=false, verbose=false)
  ```
  In the first function, `filexyz` is a string representing path of the xyz file while
  in the second function `filexyz` represent directly the content of xyz file.
  Example:
  ```julia
  # Initialize using an existing xyz file
  atoms = init_atoms_xyz("H2O.xyz")
  # Initialize using 'inline' xyz file
  atoms = init_atoms_xyz(
  """
  2

  O   -0.8  0.0  0.0
  O    0.8  0.0  0.0
  """
  ```

Note that, for both way `LatVecs` must be set manually by:
```julia
atoms.LatVecs = 16*eye(3) # for example
```
Currently there is no warning or check being performed to make sure that `LatVecs`
is defined properly. The default value is `zeros(3,3)`. So an error will happen
if an instance of `PWGrid` is constucted because we will try to invert a zero matrix.


Equation
\begin{equation}
\frac{\alpha}{\beta}
\end{equation}


\end{markdown}


\section{Plane wave basis set, real space grid, and k-points}

The type \jlinline{PWGrid} wraps various variables related to plane wave basis
set. This has two fields of type \jlinline{GVectors}
and \jlinline{GVectorsW} for storing information about $\mathbf{G}$-vectors
that are used in potential and wave functions, respectively.

\begin{figure}[H]
\centering
\begin{tikzpicture}
\node[mybox] (box) {%
\begin{minipage}{0.6\textwidth}%
\begin{minted}{julia}
struct PWGrid
    ecutwfc::Float64
    ecutrho::Float64
    Ns::Tuple{Int64,Int64,Int64}
    LatVecs::Array{Float64,2}
    RecVecs::Array{Float64,2}
    CellVolume::Float64
    r::Array{Float64,2}
    gvec::GVectors
    gvecw::GVectorsW
    planfw
    planbw
end
\end{minted}
\end{minipage}
};
\node[fancytitle, right=10pt] at (box.north west) {\jlinline{PWGrid} struct definition};
\end{tikzpicture}
\caption{Definition of \jlinline{PWGrid}. The type annotation of \jlinline{planfw} and \jlinline{planbw} is
omitted because they are too long.}
\end{figure}


We can define grid points over unit cell as:
$$
\mathbf{r} = \frac{i}{N_{s1}}\mathbf{a}_{1} + \frac{j}{N_{s2}}\mathbf{a}_{2} +
\frac{k}{N_{s3}}\mathbf{a}_{3}
$$
where $i = 0,1,\ldots,N_{s1}-1$, $j = 0,1,\ldots,N_{s2}-1$, $k = 0,1,\ldots,N_{s3}-1$

\begin{figure}[H]
\centering
\begin{tikzpicture}
\node[mybox] (box) {%
\begin{minipage}{0.6\textwidth}%
\begin{minted}[fontsize=\footnotesize]{julia}
struct GVectors
  Ng::Int64
  G::Array{Float64,2}
  G2::Array{Float64,1}
  idx_g2r::Array{Int64,1}
  G2_shells::Array{Float64,1}
  idx_g2shells::Array{Int64,1}
end
\end{minted}
\end{minipage}
};
\node[fancytitle, right=10pt] at (box.north west) {\jlinline{GVectors} struct definition};
\end{tikzpicture}
\caption{Definition of \jlinline{GVectors}.}
\end{figure}


\begin{figure}[H]
\centering
\begin{tikzpicture}
\node[mybox] (box) {%
\begin{minipage}{0.6\textwidth}%
\begin{minted}[fontsize=\footnotesize]{julia}
struct GVectorsW
  Ngwx::Int64
  Ngw::Array{Int64,1}
  idx_gw2g::Array{Array{Int64,1},1}
  idx_gw2r::Array{Array{Int64,1},1}
  kpoints::KPoints
end
\end{minted}
\end{minipage}
};
\node[fancytitle, right=10pt] at (box.north west) {\jlinline{GVectorsW} struct definition};
\end{tikzpicture}
\caption{Definition of \jlinline{GVectorsW}.}
\end{figure}


The $\mathbf{G}$-vectors can be defined as:
\begin{equation}
\mathbf{G} = n_1 \mathbf{b}_1 + n_2 \mathbf{b}_2 + n_3 \mathbf{b}_3
\end{equation}
where $n_1, n_2, n_3$ are integer numbers and
$\mathbf{b}_1, \mathbf{b}_2, \mathbf{b}_3$ are three vectors describing
unit cell of reciprocal lattice or \textit{unit reciprocal lattice vectors}.
They satisfy the following relations:
\begin{align}
\mathbf{a}_1 = 2\pi\frac{\mathbf{a}_{2} \times \mathbf{a}_{3}}{\Omega}
\mathbf{a}_2 = 2\pi\frac{\mathbf{a}_{3} \times \mathbf{a}_{1}}{\Omega}
\mathbf{a}_3 = 2\pi\frac{\mathbf{a}_{1} \times \mathbf{a}_{2}}{\Omega}
\end{align}

A periodic function
\begin{equation}
f(\mathbf{r}) = f(\mathbf{r}+\mathbf{L}),\,\,\,
\mathbf{L} = n_{1}a_{1} + n_{2}a_{2} + n_{3}a_{3}
\end{equation}
can be expanded using plane wave basis basis functions as:
\begin{equation}
f(\mathbf{r}) = \frac{1}{\sqrt{\Omega}}\sum_{\mathbf{G}}
C_{\mathbf{G}} \exp(\imath \mathbf{G} \cdot \mathbf{r})
\end{equation}
where $C_{\mathbf{G}}$ are expansion coefficients. This sum is usually truncated
at a certain maximum value of $\mathbf{G}$-vector, $\mathbf{G}_{\mathrm{max}}$.

Kohn-Sham wave function:
\begin{equation}
\psi_{i,\mathbf{k}}(\mathbf{r}) = u_{i,\mathbf{k}}(\mathbf{r}) \exp\left[ \imath \mathbf{k} \cdot \mathbf{r} \right]
\end{equation}
where $u_{i,\mathbf{k}}(\mathbf{r}) = u_{i,\mathbf{k}}(\mathbf{r}+\mathbf{L})$

Using plane wave expansion:
\begin{equation}
u_{i,\mathbf{k}}(\mathbf{r}) =
\frac{1}{\sqrt{\Omega}}\sum_{\mathbf{G}} C_{i,\mathbf{k},\mathbf{G}} \exp(\imath \mathbf{G} \cdot \mathbf{r}),
\end{equation}
%
we have:
\begin{equation}
\psi_{i,\mathbf{k}}(\mathbf{r}) =
\frac{1}{\sqrt{\Omega}}\sum_{\mathbf{G}} C_{i,\mathbf{G+\mathbf{k}}}
\exp\left[ \imath (\mathbf{G}+\mathbf{k}) \cdot \mathbf{r} \right]
\end{equation}

With this expression we can expand electronic density in plane wave basis:
\begin{align*}
\rho(\mathbf{r}) & = \sum_{i} \int f_{i,\mathbf{k}}
\psi^{*}_{i,\mathbf{k}}(\mathbf{r}) \psi_{i,\mathbf{k}}(\mathbf{r})
\,\mathrm{d}\mathbf{k} \\
%
& = \frac{1}{\Omega} \sum_{i} \int f_{i,\mathbf{k}}
\left(
\sum_{\mathbf{G}'} C_{i,\mathbf{G'+\mathbf{k}}}
\exp\left[ -\imath (\mathbf{G}'+\mathbf{k}) \cdot \mathbf{r} \right]
\right)
%
\left(
\sum_{\mathbf{G}} C_{i,\mathbf{G+\mathbf{k}}}
\exp\left[ \imath (\mathbf{G}+\mathbf{k}) \cdot \mathbf{r} \right]
\right)
\,\mathrm{d}\mathbf{k} \\
%
& = \frac{1}{\Omega} \sum_{i} \int f_{i,\mathbf{k}}
\sum_{\mathbf{G}} \sum_{\mathbf{G}'}
C_{i,\mathbf{G+\mathbf{k}}} C_{i,\mathbf{G'+\mathbf{k}}}
\exp\left[ \imath (\mathbf{G}-\mathbf{G}') \cdot \mathbf{r} \right]
\,\mathrm{d}\mathbf{k} \\
%
& = \frac{1}{\Omega} \sum_{\mathbf{G}''}
C_{\mathbf{G}''} \exp\left[ \imath \mathbf{G}'' \cdot \mathbf{r} \right]
\,\mathrm{d}\mathbf{k}
\end{align*}
The sum over $\mathbf{G}''$ extends twice the range over the range needed
by the wave function expansion.

For wave function expansion we use plane wave expansion over $\mathbf{G}$
vectors defined by:
\begin{equation}
\frac{1}{2} \left| \mathbf{G} + \mathbf{k} \right|^2 \leq E_{\mathrm{cut}}
\label{eq:ecutwfc_def}
\end{equation}
where $E_{\mathrm{cut}}$ is a given cutoff energy which corresponds
to \jlinline{ecutwfc} field of \jlinline{PWGrid}.
For electronic density (and potentials) we have:
\begin{equation}
\frac{1}{2} \mathbf{G}^2 \leq 4 E_{\mathrm{cut}}
\label{eq:ecutrho_def}
\end{equation}
The value of $4 E_{\mathrm{cut}}$ corresponds to \jlinline{ecutrho} field of
of \jlinline{PWGrid}.

In the implementation, we first generate a set of $\mathbf{G}$-vectors which satisfies
Equation \eqref{eq:ecutrho_def} and derives several subsets from it which
satisfy Equation \eqref{eq:ecutwfc_def} for a given $\mathbf{k}$-points.

An instance of \jlinline{PWGrid} can be initialized by using its constructor
which has the following signature:
\begin{juliacode}
function PWGrid( ecutwfc::Float64, LatVecs::Array{Float64,2};
    kpoints=nothing, Ns_=(0,0,0) )
\end{juliacode}
There are two mandatory arguments: \jlinline{ecutwfc} and \jlinline{LatVecs}.
\jlinline{ecutwf} is cutoff energy for kinetic energy (in Hartree) and
\jlinline{LatVecs} is usually correspond to the one used in an
instance of \jlinline{Atoms}.

Structure factor

FFT

operators op nabla op nabla 2

\documentclass{article}
\usepackage{tikz}
\usetikzlibrary{shapes.geometric}

\tikzstyle{mybox} = [draw=blue, fill=green!5, very thick, rectangle,
  rounded corners, inner sep=10pt, inner ysep=20pt]

\tikzstyle{fancytitle} = [fill=blue, text=white]

\usepackage{minted}
\newcommand{\jlinline}[1]{\mintinline{julia}{#1}}

\begin{document}

\thispagestyle{empty}

\begin{tikzpicture}

\node[mybox] (box) {%
\begin{minipage}{0.6\textwidth}%
\begin{minted}{julia}
struct KPoints
    Nkpt::Int64
    mesh::Tuple{Int64,Int64,Int64}
    k::Array{Float64,2}
    wk::Array{Float64,1}
    RecVecs::Array{Float64,2}
end
\end{minted}
\end{minipage}
};

\node[fancytitle, right=10pt] at (box.north west) {\jlinline{GVectors} struct definition};

\end{tikzpicture}

\end{document}



\section{Electronic states}

\begin{figure}[H]
\centering
\begin{tikzpicture}
\node[mybox] (box) {%
\begin{minipage}{0.6\textwidth}%
\begin{minted}{julia}
mutable struct Electrons
    Nelectrons::Float64
    Nstates::Int64
    Nstates_occ::Int64
    Focc::Array{Float64,2}
    ebands::Array{Float64,2}
    Nspin::Int64
end
\end{minted}
\end{minipage}
};
\node[fancytitle, right=10pt] at (box.north west) {\jlinline{Electrons} struct definition};
\end{tikzpicture}
\caption{Definition of \jlinline{Electrons}.}
\end{figure}



\section{Potentials and energies}

In KSDFT approach \cite{Hohenberg1964,Kohn1965}, total energy per unit cell system
$E^{\mathrm{KS}}_{\mathrm{total}}$ can be written as
\begin{equation}
E^{\mathrm{KS}}_{\mathrm{total}} =
E_{\mathrm{kin}} + E_{\mathrm{ele-nuc}} +
E_{\mathrm{Ha}} + E_{\mathrm{xc}} + E_{\mathrm{nuc-nuc}}
\label{eq:E_KS_total}
\end{equation}

Kohn-Sham equations:
\begin{equation}
H_{\mathrm{KS}} \psi_{i\mathbf{k}}(\mathbf{r}) =
\epsilon_{i\mathbf{k}} \psi_{i\mathbf{k}}(\mathbf{r})
\end{equation}

\begin{figure}[H]
\centering
\begin{tikzpicture}
\node[mybox] (box) {%
\begin{minipage}{0.6\textwidth}%
\begin{minted}{julia}
mutable struct Potentials
    Ps_loc::Array{Float64,1}
    Hartree::Array{Float64,1}
    XC::Array{Float64,2}
    Total::Array{Float64,2}
end
\end{minted}
\end{minipage}
};
\node[fancytitle, right=10pt] at (box.north west) {\jlinline{Potentials} struct definition};
\end{tikzpicture}
\caption{Definition of \jlinline{Potentials}.}
\end{figure}


\begin{figure}[H]
\centering
\begin{tikzpicture}
\node[mybox] (box) {%
\begin{minipage}{0.6\textwidth}%
\begin{minted}{julia}
mutable struct Energies
    Kinetic::Float64
    Ps_loc::Float64
    Ps_nloc::Float64
    Hartree::Float64
    XC::Float64
    NN::Float64
    PspCore::Float64
    mTS::Float64
end
\end{minted}
\end{minipage}
};
\node[fancytitle, right=10pt] at (box.north west) {\jlinline{Energies} struct definition};
\end{tikzpicture}
\caption{Definition of \jlinline{Energies}.}
\end{figure}


\subsection{Electron density}

Electron density $\rho(\mathbf{r})$ is calculated as:
\begin{equation}
\rho(\mathbf{r}) = \sum_{i=1}^{N_{\mathrm{occ}}} f_{i} \psi^{*}_{i}(\mathbf{r})
\psi_{i}(\mathbf{r})
\end{equation}

Function: \jlinline{calc_rhoe!} and \jlinline{calc_rhoe}.

\subsection{Kinetic energy}

Kinetic energy:



\subsection{Local and nonlocal pseudopotential energy}


\subsection{Hartree energy}


\subsection{XC energy and potential}

\textsf{PWDFT.jl} uses \textsf{Libxc.jl}\cite{Libxc.jl}, a Julia wrapper to
\textsf{Libxc}\cite{Marques2012,Lehtola2018}, to calculate exchange correlation
energy and potentials.

For LDA we have:
\begin{align}
E_{\mathrm{xc}}\left[\rho_{\sigma}\right] & = \int \epsilon^{\mathrm{HEG}}_{\mathrm{xc}}
\left[ \rho_{\sigma}(\mathbf{r}) \right]
\rho_{\mathrm{tot}}(\mathbf{r})\, \mathrm{d}\mathbf{r} \\
& = \int \left\{
\epsilon^{\mathrm{HEG}}_{\mathrm{x}} \left[ \rho_{\sigma}(\mathbf{r}) \right] +
\epsilon^{\mathrm{HEG}}_{\mathrm{c}} \left[ \rho_{\sigma}(\mathbf{r}) \right]
\right\}
\rho(\mathbf{r})\, \mathrm{d}\mathbf{r}
\end{align}

\begin{equation}
\delta E_{\mathrm{xc}}\left[\rho_{\sigma}\right] =
\sum_{\sigma} \int
\left(
\epsilon^{\mathrm{HEG}}_{\mathrm{xc}} +
\rho_{\mathrm{tot}} \frac{\partial}{\partial \rho_{\sigma}} \epsilon^{\mathrm{HEG}}_{\mathrm{xc}}
\right)
\, \mathrm{d}\mathbf{r}\,\delta \rho_{\sigma}
\end{equation}


\subsubsection{Calculation of $E_{\mathrm{xc}}$ in \textsf{PWDFT.jl} using Libxc}

Note:

For VWN functional (should be applicable to other LDA functionals), we have the following
for non-spin-polarized case:
%
\begin{juliacode}
function calc_epsxc_VWN( Rhoe::Array{Float64,1} )
    Npoints = size(Rhoe)[1]
    Nspin = 1
    eps_x = zeros(Float64,Npoints)
    eps_c = zeros(Float64,Npoints)

    ptr = Libxc.xc_func_alloc()

    # exchange part
    Libxc.xc_func_init(ptr, Libxc.LDA_X, Nspin)
    Libxc.xc_lda_exc!(ptr, Npoints, Rhoe, eps_x)
    Libxc.xc_func_end(ptr)

    # correlation part
    Libxc.xc_func_init(ptr, Libxc.LDA_C_VWN, Nspin)
    Libxc.xc_lda_exc!(ptr, Npoints, Rhoe, eps_c)
    Libxc.xc_func_end(ptr)

    Libxc.xc_func_free(ptr)

    return eps_x + eps_c
end
\end{juliacode}


\begin{juliacode}
function calc_epsxc_VWN( Rhoe::Array{Float64,2} )
    Nspin = size(Rhoe)[2]
    Npoints = size(Rhoe)[1]
    if Nspin == 1
        return calc_epsxc_VWN( Rhoe[:,1] )
    end

    # Do the transpose manually
    Rhoe_tmp = zeros(2*Npoints)
    ipp = 0
    for ip = 1:2:2*Npoints
        ipp = ipp + 1
        Rhoe_tmp[ip] = Rhoe[ipp,1]
        Rhoe_tmp[ip+1] = Rhoe[ipp,2]
    end

    # ....
    # The rest of the code are similar to non-spin polarized case, however now
    # we use `Nspin=2` and pass `Rhoe_tmp` instead of `Rhoe`
end
\end{juliacode}

For PBE:
\begin{juliacode}
function calc_epsxc_PBE( pw::PWGrid, Rhoe::Array{Float64,1} )
    Npoints = size(Rhoe)[1]
    Nspin = 1

    # calculate gRhoe2
    gRhoe = op_nabla( pw, Rhoe )
    gRhoe2 = zeros( Float64, Npoints )
    for ip = 1:Npoints
        gRhoe2[ip] = dot( gRhoe[:,ip], gRhoe[:,ip] )
    end

    eps_x = zeros(Float64,Npoints)
    eps_c = zeros(Float64,Npoints)

    ptr = Libxc.xc_func_alloc()

    # exchange part
    Libxc.xc_func_init(ptr, Libxc.GGA_X_PBE, Nspin)
    Libxc.xc_gga_exc!(ptr, Npoints, Rhoe, gRhoe2, eps_x)
    Libxc.xc_func_end(ptr)

    # correlation part
    Libxc.xc_func_init(ptr, Libxc.GGA_C_PBE, Nspin)
    Libxc.xc_gga_exc!(ptr, Npoints, Rhoe, gRhoe2, eps_c)
    Libxc.xc_func_end(ptr)

    Libxc.xc_func_free(ptr)

    return eps_x + eps_c
end
\end{juliacode}

For PBE spin-polarized case

\begin{juliacode}
function calc_epsxc_PBE( pw::PWGrid, Rhoe::Array{Float64,2} )
    Nspin = size(Rhoe)[2]
    if Nspin == 1
        return calc_epsxc_PBE( pw, Rhoe[:,1] )
    end

    Npoints = size(Rhoe)[1]

    # calculate gRhoe2
    gRhoe_up = op_nabla( pw, Rhoe[:,1] )
    gRhoe_dn = op_nabla( pw, Rhoe[:,2] )
    gRhoe2 = zeros( Float64, 3*Npoints )
    ipp = 0
    for ip = 1:3:3*Npoints
        ipp = ipp + 1
        gRhoe2[ip]   = dot( gRhoe_up[:,ipp], gRhoe_up[:,ipp] )
        gRhoe2[ip+1] = dot( gRhoe_up[:,ipp], gRhoe_dn[:,ipp] )
        gRhoe2[ip+2] = dot( gRhoe_dn[:,ipp], gRhoe_dn[:,ipp] )
    end

    Rhoe_tmp = zeros(2*Npoints)
    ipp = 0
    for ip = 1:2:2*Npoints
        ipp = ipp + 1
        Rhoe_tmp[ip] = Rhoe[ipp,1]
        Rhoe_tmp[ip+1] = Rhoe[ipp,2]
    end

    # ....
    # The rest of the code are similar to non-spin polarized case, however now
    # we use `Nspin=2` and pass `Rhoe_tmp` instead of `Rhoe`
end
\end{juliacode}


\subsubsection{Calculation of $V_{\mathrm{xc}}$ in \textsf{PWDFT.jl} using Libxc}

VWN non-spin polarized:
\begin{juliacode}
function calc_Vxc_VWN( Rhoe::Array{Float64,1} )
    Npoints = size(Rhoe)[1]
    Nspin = 1
    v_x = zeros(Float64,Npoints)
    v_c = zeros(Float64,Npoints)

    ptr = Libxc.xc_func_alloc()

    # exchange part
    Libxc.xc_func_init(ptr, Libxc.LDA_X, Nspin)
    Libxc.xc_lda_vxc!(ptr, Npoints, Rhoe, v_x)
    Libxc.xc_func_end(ptr)

    # correlation part
    Libxc.xc_func_init(ptr, Libxc.LDA_C_VWN, Nspin)
    Libxc.xc_lda_vxc!(ptr, Npoints, Rhoe, v_c)
    Libxc.xc_func_end(ptr)

    Libxc.xc_func_free(ptr)

    return v_x + v_c
end
\end{juliacode}

VWN spin-polarized:
\begin{juliacode}
function calc_Vxc_VWN( Rhoe::Array{Float64,2} )
    Nspin = size(Rhoe)[2]
    if Nspin == 1
        return calc_Vxc_VWN( Rhoe[:,1] )
    end

    Npoints = size(Rhoe)[1]

    Vxc = zeros( Float64, Npoints, 2 )
    V_x = zeros( Float64, 2*Npoints )
    V_c = zeros( Float64, 2*Npoints )

    # This is the transposed version of Rhoe, use copy
    Rhoe_tmp = zeros(2*Npoints)
    ipp = 0
    for ip = 1:2:2*Npoints
        ipp = ipp + 1
        Rhoe_tmp[ip] = Rhoe[ipp,1]
        Rhoe_tmp[ip+1] = Rhoe[ipp,2]
    end

    ptr = Libxc.xc_func_alloc()

    # exchange part
    Libxc.xc_func_init(ptr, Libxc.LDA_X, Nspin)
    Libxc.xc_lda_vxc!(ptr, Npoints, Rhoe_tmp, V_x)
    Libxc.xc_func_end(ptr)

    # correlation part
    Libxc.xc_func_init(ptr, Libxc.LDA_C_VWN, Nspin)
    Libxc.xc_lda_vxc!(ptr, Npoints, Rhoe_tmp, V_c)
    Libxc.xc_func_end(ptr)

    Libxc.xc_func_free(ptr)

    ipp = 0
    for ip = 1:2:2*Npoints
        ipp = ipp + 1
        Vxc[ipp,1] = V_x[ip] + V_c[ip]
        Vxc[ipp,2] = V_x[ip+1] + V_c[ip+1]
    end
    return Vxc
end
\end{juliacode}

PBE non-spin-polarized:
\begin{juliacode}
function calc_Vxc_PBE( pw::PWGrid, Rhoe::Array{Float64,1} )
    Npoints = size(Rhoe)[1]
    Nspin = 1

    # calculate gRhoe2
    gRhoe = op_nabla( pw, Rhoe )
    gRhoe2 = zeros( Float64, Npoints )
    for ip = 1:Npoints
        gRhoe2[ip] = dot( gRhoe[:,ip], gRhoe[:,ip] )
    end

    V_x = zeros(Float64,Npoints)
    V_c = zeros(Float64,Npoints)
    V_xc = zeros(Float64,Npoints)

    Vg_x = zeros(Float64,Npoints)
    Vg_c = zeros(Float64,Npoints)
    Vg_xc = zeros(Float64,Npoints)

    ptr = Libxc.xc_func_alloc()

    # exchange part
    Libxc.xc_func_init(ptr, Libxc.GGA_X_PBE, Nspin)
    Libxc.xc_gga_vxc!(ptr, Npoints, Rhoe, gRhoe2, V_x, Vg_x)
    Libxc.xc_func_end(ptr)

    # correlation part
    Libxc.xc_func_init(ptr, Libxc.GGA_C_PBE, Nspin)
    Libxc.xc_gga_vxc!(ptr, Npoints, Rhoe, gRhoe2, V_c, Vg_c)
    Libxc.xc_func_end(ptr)

    V_xc = V_x + V_c
    Vg_xc = Vg_x + Vg_c

    # gradient correction
    h = zeros(Float64,3,Npoints)
    divh = zeros(Float64,Npoints)
    for ip = 1:Npoints
        h[1,ip] = Vg_xc[ip] * gRhoe[1,ip]
        h[2,ip] = Vg_xc[ip] * gRhoe[2,ip]
        h[3,ip] = Vg_xc[ip] * gRhoe[3,ip]
    end
    # div ( vgrho * gRhoe )
    divh = op_nabla_dot( pw, h )
    for ip = 1:Npoints
        V_xc[ip] = V_xc[ip] - 2.0*divh[ip]
    end

    return V_xc
end
\end{juliacode}

PBE spin-polarized:
\begin{juliacode}
function calc_Vxc_PBE( pw::PWGrid, Rhoe::Array{Float64,2} )
    Nspin = size(Rhoe)[2]
    if Nspin == 1
        return calc_Vxc_PBE( pw, Rhoe[:,1] )
    end

    Npoints = size(Rhoe)[1]

    # calculate gRhoe2
    gRhoe_up = op_nabla( pw, Rhoe[:,1] ) # gRhoe = ∇⋅Rhoe
    gRhoe_dn = op_nabla( pw, Rhoe[:,2] )
    gRhoe2 = zeros( Float64, 3*Npoints )
    ipp = 0
    for ip = 1:3:3*Npoints
        ipp = ipp + 1
        gRhoe2[ip]   = dot( gRhoe_up[:,ipp], gRhoe_up[:,ipp] )
        gRhoe2[ip+1] = dot( gRhoe_up[:,ipp], gRhoe_dn[:,ipp] )
        gRhoe2[ip+2] = dot( gRhoe_dn[:,ipp], gRhoe_dn[:,ipp] )
    end

    V_xc = zeros(Float64, Npoints, 2)
    V_x  = zeros(Float64, Npoints*2)
    V_c  = zeros(Float64, Npoints*2)

    Vg_xc = zeros(Float64, 3, Npoints)
    Vg_x  = zeros(Float64, 3*Npoints)
    Vg_c  = zeros(Float64, 3*Npoints)

    Rhoe_tmp = zeros(2*Npoints)
    ipp = 0
    for ip = 1:2:2*Npoints
        ipp = ipp + 1
        Rhoe_tmp[ip] = Rhoe[ipp,1]
        Rhoe_tmp[ip+1] = Rhoe[ipp,2]
    end

    ptr = Libxc.xc_func_alloc()

    # exchange part
    Libxc.xc_func_init(ptr, Libxc.GGA_X_PBE, Nspin)
    Libxc.xc_gga_vxc!(ptr, Npoints, Rhoe_tmp, gRhoe2, V_x, Vg_x)
    Libxc.xc_func_end(ptr)

    # correlation part
    Libxc.xc_func_init(ptr, Libxc.GGA_C_PBE, Nspin)
    Libxc.xc_gga_vxc!(ptr, Npoints, Rhoe_tmp, gRhoe2, V_c, Vg_c)
    Libxc.xc_func_end(ptr)

    ipp = 0
    for ip = 1:2:2*Npoints
        ipp = ipp + 1
        V_xc[ipp,1] = V_x[ip] + V_c[ip]
        V_xc[ipp,2] = V_x[ip+1] + V_c[ip+1]
    end

    Vg_xc = reshape(Vg_x + Vg_c, (3,Npoints))

    h = zeros(Float64,3,Npoints)
    divh = zeros(Float64,Npoints)

    # spin up
    for ip = 1:Npoints
        h[1,ip] = 2*Vg_xc[1,ip]*gRhoe_up[1,ip] + Vg_xc[2,ip]*gRhoe_dn[1,ip]
        h[2,ip] = 2*Vg_xc[1,ip]*gRhoe_up[2,ip] + Vg_xc[2,ip]*gRhoe_dn[2,ip]
        h[3,ip] = 2*Vg_xc[1,ip]*gRhoe_up[3,ip] + Vg_xc[2,ip]*gRhoe_dn[3,ip]
    end
    divh = op_nabla_dot( pw, h )
    for ip = 1:Npoints
        V_xc[ip,1] = V_xc[ip,1] - divh[ip]
    end

    # Spin down
    for ip = 1:Npoints
        h[1,ip] = 2*Vg_xc[3,ip]*gRhoe_dn[1,ip] + Vg_xc[2,ip]*gRhoe_up[1,ip]
        h[2,ip] = 2*Vg_xc[3,ip]*gRhoe_dn[2,ip] + Vg_xc[2,ip]*gRhoe_up[2,ip]
        h[3,ip] = 2*Vg_xc[3,ip]*gRhoe_dn[3,ip] + Vg_xc[2,ip]*gRhoe_up[3,ip]
    end
    divh = op_nabla_dot( pw, h )
    for ip = 1:Npoints
        V_xc[ip,2] = V_xc[ip,2] - divh[ip]
    end

    return V_xc
end
\end{juliacode}


\section{Pseudopotentials}
%
Currently, \textsf{PWDFT.jl} supports a subset of GTH (Goedecker-Teter-Hutter)
pseudopotentials. This type of pseudopotential is analytic and thus is somewhat
easier to program.
%
\textsf{PWDFT.jl} distribution contains several parameters
of GTH pseudopotentials for LDA and GGA functionals.

These pseudopotentials can be written in terms of
local $V^{\mathrm{PS}}_{\mathrm{loc}}$ and
angular momentum $l$ dependent
nonlocal components $\Delta V^{\mathrm{PS}}_{l}$:
\begin{equation}
V_{\mathrm{ene-nuc}}(\mathbf{r}) =
\sum_{I} \left[
V^{\mathrm{PS}}_{\mathrm{loc}}(\mathbf{r}-\mathbf{R}_{I}) +
\sum_{l=0}^{l_{\mathrm{max}}}
V^{\mathrm{PS}}_{l}(\mathbf{r}-\mathrm{R}_{I},\mathbf{r}'-\mathbf{R}_{I})
\right]
\end{equation}

\subsection{Local pseudopotential}

The local pseudopotential for
$I$-th atom, $V^{\mathrm{PS}}_{\mathrm{loc}}(\mathbf{r}-\mathbf{R}_{I})$,
is radially symmetric
function with the following radial form
\begin{equation}
V^{\mathrm{PS}}_{\mathrm{loc}}(r) =
-\frac{Z_{\mathrm{val}}}{r}\mathrm{erf}\left[
\frac{\bar{r}}{\sqrt{2}} \right] +
\exp\left[-\frac{1}{2}\bar{r}^2\right]\left(
C_{1} + C_{2}\bar{r}^2 + C_{3}\bar{r}^4 + C_{4}\bar{r}^6
\right)
\label{eq:V_ps_loc_R}
\end{equation}
with $\bar{r}=r/r_{\mathrm{loc}}$ and $r_{\mathrm{loc}}$, $Z_{\mathrm{val}}$,
$C_{1}$, $C_{2}$, $C_{3}$ and $C_{4}$ are the corresponding pseudopotential
parameters.
In $\mathbf{G}$-space, the GTH local pseudopotential can be written as
\begin{multline}
V^{\mathrm{PS}}_{\mathrm{loc}}(G) = -\frac{4\pi}{\Omega}\frac{Z_{\mathrm{val}}}{G^2}
\exp\left[-\frac{x^2}{2}\right] +
\sqrt{8\pi^3} \frac{r^{3}_{\mathrm{loc}}}{\Omega}\exp\left[-\frac{x^2}{2}\right]\times\\
\left( C_{1} + C_{2}(3 - x^2) + C_{3}(15 - 10x^2 + x^4) + C_{4}(105 - 105x^2 + 21x^4 - x^6) \right)
\label{eq:V_ps_loc_G}
\end{multline}
where $x=G r_{\mathrm{loc}}$.

\subsection{Nonlocal pseudopotential}

The nonlocal component of GTH pseudopotential can written in real space as
\begin{equation}
V^{\mathrm{PS}}_{l}(\mathbf{r}-\mathbf{R}_{I},\mathbf{r}'-\mathbf{R}_{I}) =
\sum_{\mu=1}^{N_{l}} \sum_{\nu=1}^{N_{l}} \sum_{m=-l}^{l}
\beta_{\mu lm}(\mathbf{r}-\mathbf{R}_{I})\,
h^{l}_{\mu\nu}\,
\beta^{*}_{\nu lm}(\mathbf{r}'-\mathbf{R}_{I})
\end{equation}
where $\beta_{\mu lm}(\mathbf{r})$ are atomic-centered projector functions
\begin{equation}
\beta_{\mu lm}(\mathbf{r}) =
p^{l}_{\mu}(r) Y_{lm}(\hat{\mathbf{r}})
\label{eq:proj_NL_R}
\end{equation}
%
and $h^{l}_{\mu\nu}$ are the pseudopotential parameters and
$Y_{lm}$ are the spherical harmonics. Number of projectors per angular
momentum $N_{l}$ may take value up to 3 projectors.
%
In $\mathbf{G}$-space, the nonlocal part of GTH pseudopotential can be described by
the following equation.
\begin{equation}
V^{\mathrm{PS}}_{l}(\mathbf{G},\mathbf{G}') =
(-1)^{l} \sum_{\mu}^{3} \sum_{\nu}^{3}\sum_{m=-l}^{l}
\beta_{\mu l m}(\mathbf{G}) h^{l}_{\mu\nu}
\beta^{*}_{\nu l m}(\mathbf{G}')
\end{equation}
with the projector functions
\begin{equation}
\beta_{\mu lm}(\mathbf{G}) = p^{l}_{\mu}(G) Y_{lm}(\hat{\mathbf{G}})
\label{eq:betaNL_G}
\end{equation}
The radial part of projector functions take the following form
\begin{equation}
p^{l}_{\mu}(G) = q^{l}_{\mu}\left(Gr_{l}\right)
\frac{\pi^{5/4}G^{l}\sqrt{ r_{l}^{2l+3}}}{\sqrt{\Omega}}
\exp\left[-\frac{1}{2}G^{2}r^{2}_{l}\right]
\label{eq:proj_NL_G}
\end{equation}
%
For $l=0$, we consider up to $N_{l}=3$ projectors:
\begin{align}
q^{0}_{1}(x) & = 4\sqrt{2} \\
q^{0}_{2}(x) & = 8\sqrt{\frac{2}{15}}(3 - x^2) \\
q^{0}_{3}(x) & = \frac{16}{3}\sqrt{\frac{2}{105}} (15 - 20x^2 + 4x^4)
\end{align}
%
For $l=1$, we consider up to $N_{l}=3$ projectors:
\begin{align}
q^{1}_{1}(x) & = 8 \sqrt{\frac{1}{3}} \\
q^{1}_{2}(x) & = 16 \sqrt{\frac{1}{105}} (5 - x^2) \\
q^{1}_{3}(x) & = 8 \sqrt{\frac{1}{1155}} (35 - 28x^2 + 4x^4)
\end{align}
%
For $l=2$, we consider up to $N_{l}=2$ projectors:
\begin{align}
q^{2}_{1}(x) & = 8\sqrt{\frac{2}{15}} \\
q^{2}_{2}(x) & = \frac{16}{3} \sqrt{\frac{2}{105}}(7 - x^2)
\end{align}
%
For $l=3$, we only consider up to $N_{l}=1$ projector:
\begin{equation}
q^{3}_{1}(x) = 16\sqrt{\frac{1}{105}}
\end{equation}

In the present implementation, we construct the local and nonlocal
components of pseudopotential in the $\mathbf{G}$-space using
their Fourier-transformed expressions
and transformed them back to real space if needed.
We refer the readers to the original
reference \cite{Goedecker1996} and the book \cite{Marx2009}
for more information about GTH pseudopotentials.

Due to the separation of local and non-local components of electrons-nuclei
interaction, Equation \eqref{eq:E_ele_nuc} can be written as
\begin{equation}
E_{\mathrm{ele-nuc}} = E^{\mathrm{PS}}_{\mathrm{loc}}
+ E^{\mathrm{PS}}_{\mathrm{nloc}}
\end{equation}
%
The local pseudopotential contribution is
\begin{equation}
E^{\mathrm{PS}}_{\mathrm{loc}} =
\int_{\Omega} \rho(\mathbf{r})\,V^{\mathrm{PS}}_{\mathrm{loc}}(\mathbf{r})\,
\mathrm{d}\mathbf{r}
\end{equation}
%
and the non-local contribution is
\begin{equation}
E^{\mathrm{PS}}_{\mathrm{nloc}} =
\sum_{\mathbf{k}}
\sum_{i}
w_{\mathbf{k}}
f_{i\mathbf{k}}
\int_{\Omega}\,
\psi^{*}_{i\mathbf{k}}(\mathbf{r})
\left[
\sum_{I}\sum_{l=0}^{l_{\mathrm{max}}}
V^{\mathrm{PS}}_{l}(\mathbf{r}-\mathbf{R}_{I},\mathbf{r}'-\mathbf{R}_{I})
\right]
\psi_{i\mathbf{k}}(\mathbf{r})
\,\mathrm{d}\mathbf{r}.
\end{equation}


\section{Hamiltonian operators}

\begin{figure}[H]
\centering
\begin{tikzpicture}
\node[mybox] (box) {%
\begin{minipage}{0.6\textwidth}%
\begin{minted}{julia}
mutable struct Hamiltonian
    pw::PWGrid
    potentials::Potentials
    energies::Energies
    rhoe::Array{Float64,2}
    electrons::Electrons
    atoms::Atoms
    sym_info::SymmetryInfo
    rhoe_symmetrizer::RhoeSymmetrizer
    pspots::Array{PsPot_GTH,1}
    pspotNL::PsPotNL
    xcfunc::String
    ik::Int64
    ispin::Int64
end
\end{minted}
\end{minipage}
};
\node[fancytitle, right=10pt] at (box.north west) {\jlinline{Hamiltonian} struct definition};
\end{tikzpicture}
\caption{Definition of \jlinline{Hamiltonian}.}
\end{figure}


Operators:
\begin{itemize}
\item \jlinline{op_H}
\item \jlinline{op_K}
\item \jlinline{op_V_loc}
\item \jlinline{op_V_Ps_loc}
\item \jlinline{op_V_Ps_nloc}
\end{itemize}

\subsection{Iterative diagonalization of Hamiltonian}

\begin{itemize}
\item \jlinline{diag_LOBPCG}
\item \jlinline{diag_davidson}
\item \jlinline{diag_Emin_PCG}
\end{itemize}


\section{Self-consistent field}

Density vs potential mix

\jlinline{KS_solve_SCF}

\jlinline{KS_solve_SCF_potmix}

In the mean time, they are separated. They might be combined into one function
in the future development.

Mixing algorithms:
\begin{itemize}
\item Simple or linear mixing
\item Adaptive linear mixing
\item Anderson mixing
\item Broyden mixing
\item Pulay mixing
\item Restarted Pulay mixing
\item Periodic Pulay mixing
\end{itemize}


\section{Direct minimization}

\jlinline{KS_solve_Emin_PCG}


\appendix
\section{Howtos}

This part contains miscellaneous info.

\subsection*{Initializing \jlcode{Atoms}}

From xyz file: supply the path to xyz file as string and
set the lattice vectors:

\begin{juliacode}
atoms = Atoms(xyz_file="file.xyz", LatVecs=gen_lattice_sc(16.0))
\end{juliacode}

In extended xyz file, the lattice vectors information is included
(along with several others information, if any):

\begin{juliacode}
atoms = Atoms(ext_xyz_file="file.xyz")
\end{juliacode}

For crystalline systems, using keyword argument \jlcode{xyz_string_frac}
is sometimes convenient:

\begin{juliacode}
atoms = Atoms(xyz_string_frac=
        """
        2

        Si  0.0  0.0  0.0
        Si  0.25  0.25  0.25
        """, in_bohr=true,
        LatVecs=gen_lattice_fcc(10.2631))
\end{juliacode}

\textbf{IMPORTANT} We need to be careful to also specify \jlcode{in_bohr} keyword to get
the correct coordinates in bohr (which is used internally in \jlcode{PWDFT.jl}).



\subsection*{Referring or including files in \jlcode{sandbox} (or other dirs in \jlcode{PWDFT.jl})}

\begin{juliacode}
using PWDFT
const DIR_PWDFT = joinpath(dirname(pathof(PWDFT)),"..")
const DIR_PSP = joinpath(DIR_PWDFT,"pseudopotentials","pade_gth")
const DIR_STRUCTURES = joinpath(DIR_PWDFT, "structures")

pspfiles = [joinpath(DIR_PSP,"Ag-q11.gth")]
\end{juliacode}


\subsection*{Generating lattice vectors}

Lattice vectors are simply 3x3 array. We can set it manually or use
one of functions defined in \jlcode{gen_lattice_pwscf.jl}.
for generating lattice vectors for Bravais lattices that used
in Quantum ESPRESSO's PWSCF.

\subsection*{Using Babel to generate xyz file from SMILES}

\begin{textcode}
babel file.smi file.sdf
babel file.sdf file.xyz
\end{textcode}

Use \myverbcode{babel -h} to autogenerate hydrogens.



\subsection*{Setting up pseudopotentials}

One can use the function \jlcode{get_default_psp(::Atoms)} to get default
pseudopotentials set for a given instance of \jlcode{Atoms}.

Currently, it is not part of main \jlcode{PWDFT.jl} package. It is located
under \myverbcode{sandbox} subdirectory of \jlcode{PWDFT.jl} distribution.

\begin{juliacode}
using PWDFT

DIR_PWDFT = jointpath(dirname(pathof(PWDFT)),"..")
include(jointpath(DIRPWDFT,"sandbox","get_default_psp.jl"))

atoms = Atoms(ext_xyz_file="atoms.xyz")
pspfiles = get_default_psp(atoms)
\end{juliacode}

Alternatively, one can set \myverbcode{pspfiles} manually because it is simply
an array of \jlcode{String}:
\begin{juliacode}
pspfiles = ["Al-q3.gth", "O-q6.gth"]
\end{juliacode}

\textbf{IMPORTANT} Be careful to set the order of species to be same as
\jlcode{atoms.SpeciesSymbols}. For example, if
\begin{juliacode}
atoms.SpeciesSymbols = ["Al", "O", "H"]
\end{juliacode}
then
\begin{juliacode}
pspfiles = ["Al-q3.gth", "O-q6.gth", "H-q1.gth"]
\end{juliacode}

\subsection*{Initializing Hamiltonian}

For molecular systems:
\begin{juliacode}
Ham = Hamiltonian( atoms, pspfiles, ecutwfc )
\end{juliacode}

For insulator and semiconductor solids:
\begin{juliacode}
Ham = Hamiltonian( atoms, pspfiles, ecutwfc, meshk=[3,3,3] )
\end{juliacode}

For metallic systems:
\begin{juliacode}
Ham = Hamiltonian( atoms, pspfiles, ecutwfc, meshk=[3,3,3], extra_states=4 )
\end{juliacode}

Empty extra states can be specified by using \jlcode{extra_states} keyword.

For spin-polarized systems, \jlcode{Nspin} keyword can be used.


\subsection*{Iterative diagonalization of Hamiltonian}

\begin{juliacode}
evals =  diag_LOBPCG!( Ham, psiks, verbose=false, verbose_last=false,
                       Nstates_conv=Nstates_occ )
\end{juliacode}


\subsection*{Calculating electron density}

Several ways:
\begin{juliacode}
Rhoe = calc_rhoe( Nelectrons, pw, Focc, psiks, Nspin )
# or
Rhoe = calc_rhoe( Ham, psiks )
# or
calc_rhoe!( Ham, psiks, Rhoe )
\end{juliacode}

\subsection*{Read and write array (binary file)}

Write to binary files:
\begin{juliacode}
for ikspin = 1:Nkpt*Nspin
    wfc_file = open("WFC_ikspin_"*string(ikspin)*".data","w")
    write( wfc_file, psiks[ikspin] )
    close( wfc_file )
end
\end{juliacode}

Read from binary files:
\begin{juliacode}
psiks = BlochWavefunc(undef,Nkpt)
for ispin = 1:Nspin
for ik = 1:Nkpt
    ikspin = ik + (ispin-1)*Nkpt
    # Don't forget to use read mode
    wfc_file = open("WFC_ikspin_"*string(ikspin)*".data","r")
    psiks[ikspin] = Array{ComplexF64}(undef,Ngw[ik],Nstates)
    psiks[ikspin] = read!( wfc_file, psiks[ikspin] )
    close( wfc_file )
end
end
\end{juliacode}




\subsection*{Subspace rotation}

In case need sorting:
\begin{juliacode}
Hr = psiks[ikspin]' * op_H( Ham, psiks[ikspin] )
evals, evecs = eigen(Hr)
evals = real(evals[:])

# Sort in ascending order based on evals 
idx_sorted = sortperm(evals)

# Copy to Hamiltonian
Ham.electrons.ebands[:,ikspin] = evals[idx_sorted]

# and rotate
psiks[ikspin] = psiks[ikspin]*evecs[:,idx_sorted]
\end{juliacode}

Usually we don't need to sort the eigenvalues if we use Hermitian matrix. We can calculate the
subspace Hamiltonian by:
\begin{juliacode}
evals, evecs = eigen(Hermitian(Hr))
\end{juliacode}



\section*{Status}

\textbf{29 July 2019} Total energy results are now similar to ABINIT
and Quantum ESPRESSO. A rather comprehensive test has been added
for SCF and Emin PCG for several simple systems.


\textbf{28 May 2018} The following features are working now:
\begin{itemize}
\item LDA and GGA, spin-paired and spin polarized calculations
\item Calculation with k-points (for periodic solids).
  \textsf{SPGLIB} is used to reduce the Monkhorst-Pack grid points
  for integration over Brillouin zone.
\end{itemize}

Band structure calculation is possible in principle as this can be
done by simply solving
Schrodinger equation with converged Kohn-Sham potentials, however there
is currently no tidy script or function to do that.

Total energy result for isolated systems (atoms and molecules) agrees quite
well with ABINIT and PWSCF results.

\sout{Total energy result for periodic solid is quite different from ABINIT and PWSCF.
I suspect that this is related to treatment of electrostatic terms in periodic system.}

These discrepancies have been minimized. For several systems the agreement is very good
even though I did not use the same algorithm as ABINIT.

\sout{SCF is rather shaky for several systems, however it is working in quite well in nonmetallic
system.}

SCF stability has been improved with Pulay mixing and its variants.



\bibliographystyle{unsrt}
\bibliography{BIBLIO}



\end{document}
