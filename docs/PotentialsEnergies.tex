\section{Potentials and energies}

In KSDFT approach \cite{Hohenberg1964,Kohn1965}, total energy per unit cell system
$E^{\mathrm{KS}}_{\mathrm{total}}$ can be written as
\begin{equation}
E^{\mathrm{KS}}_{\mathrm{total}} =
E_{\mathrm{kin}} + E_{\mathrm{ele-nuc}} +
E_{\mathrm{Ha}} + E_{\mathrm{xc}} + E_{\mathrm{nuc-nuc}}
\label{eq:E_KS_total}
\end{equation}

Kohn-Sham equations:
\begin{equation}
H_{\mathrm{KS}} \psi_{i\mathbf{k}}(\mathbf{r}) =
\epsilon_{i\mathbf{k}} \psi_{i\mathbf{k}}(\mathbf{r})
\end{equation}

\begin{figure}[H]
\centering
\begin{tikzpicture}
\node[mybox] (box) {%
\begin{minipage}{0.6\textwidth}%
\begin{minted}{julia}
mutable struct Potentials
    Ps_loc::Array{Float64,1}
    Hartree::Array{Float64,1}
    XC::Array{Float64,2}
    Total::Array{Float64,2}
end
\end{minted}
\end{minipage}
};
\node[fancytitle, right=10pt] at (box.north west) {\jlinline{Potentials} struct definition};
\end{tikzpicture}
\caption{Definition of \jlinline{Potentials}.}
\end{figure}


\begin{figure}[H]
\centering
\begin{tikzpicture}
\node[mybox] (box) {%
\begin{minipage}{0.6\textwidth}%
\begin{minted}{julia}
mutable struct Energies
    Kinetic::Float64
    Ps_loc::Float64
    Ps_nloc::Float64
    Hartree::Float64
    XC::Float64
    NN::Float64
    PspCore::Float64
    mTS::Float64
end
\end{minted}
\end{minipage}
};
\node[fancytitle, right=10pt] at (box.north west) {\jlinline{Energies} struct definition};
\end{tikzpicture}
\caption{Definition of \jlinline{Energies}.}
\end{figure}


\seubsection{Electron density}

Electron density $\rho(\mathbf{r})$ is calculated as:
\begin{equation}
\rho(\mathbf{r}) = \sum_{i=1}^{N_{\mathrm{occ}}} f_{i} \psi^{*}_{i}(\mathbf{r})
\psi_{i}(\mathbf{r})
\end{equation}

Function: \jlinline{calc_rhoe!} and \jlinline{calc_rhoe}.

\subsection{Kinetic energy}

Kinetic energy:



\subsection{Local and nonlocal pseudopotential energy}


\subsection{Hartree energy}


\subsection{XC energy and potential}

\textsf{PWDFT.jl} uses \textsf{Libxc.jl}\cite{Libxc.jl}, a Julia wrapper to
\textsf{Libxc}\cite{Marques2012,Lehtola2018}, to calculate exchange correlation
energy and potentials.

For LDA we have:
\begin{align}
E_{\mathrm{xc}}\left[\rho_{\sigma}\right] & = \int \epsilon^{\mathrm{HEG}}_{\mathrm{xc}}
\left[ \rho_{\sigma}(\mathbf{r}) \right]
\rho_{\mathrm{tot}}(\mathbf{r})\, \mathrm{d}\mathbf{r} \\
& = \int \left\{
\epsilon^{\mathrm{HEG}}_{\mathrm{x}} \left[ \rho_{\sigma}(\mathbf{r}) \right] +
\epsilon^{\mathrm{HEG}}_{\mathrm{c}} \left[ \rho_{\sigma}(\mathbf{r}) \right]
\right\}
\rho(\mathbf{r})\, \mathrm{d}\mathbf{r}
\end{align}

\begin{equation}
\delta E_{\mathrm{xc}}\left[\rho_{\sigma}\right] =
\sum_{\sigma} \int
\left(
\epsilon^{\mathrm{HEG}}_{\mathrm{xc}} +
\rho_{\mathrm{tot}} \frac{\partial}{\partial \rho_{\sigma}} \epsilon^{\mathrm{HEG}}_{\mathrm{xc}}
\right)
\, \mathrm{d}\mathbf{r}\,\delta \rho_{\sigma}
\end{equation}


\subsubsection{Calculation of $E_{\mathrm{xc}}$ in \textsf{PWDFT.jl} using Libxc}

Note:

For VWN functional (should be applicable to other LDA functionals), we have the following
for non-spin-polarized case:
%
\begin{juliacode}
function calc_epsxc_VWN( Rhoe::Array{Float64,1} )
    Npoints = size(Rhoe)[1]
    Nspin = 1
    eps_x = zeros(Float64,Npoints)
    eps_c = zeros(Float64,Npoints)

    ptr = Libxc.xc_func_alloc()

    # exchange part
    Libxc.xc_func_init(ptr, Libxc.LDA_X, Nspin)
    Libxc.xc_lda_exc!(ptr, Npoints, Rhoe, eps_x)
    Libxc.xc_func_end(ptr)

    # correlation part
    Libxc.xc_func_init(ptr, Libxc.LDA_C_VWN, Nspin)
    Libxc.xc_lda_exc!(ptr, Npoints, Rhoe, eps_c)
    Libxc.xc_func_end(ptr)

    Libxc.xc_func_free(ptr)

    return eps_x + eps_c
end
\end{juliacode}


\begin{juliacode}
function calc_epsxc_VWN( Rhoe::Array{Float64,2} )
    Nspin = size(Rhoe)[2]
    Npoints = size(Rhoe)[1]
    if Nspin == 1
        return calc_epsxc_VWN( Rhoe[:,1] )
    end

    # Do the transpose manually
    Rhoe_tmp = zeros(2*Npoints)
    ipp = 0
    for ip = 1:2:2*Npoints
        ipp = ipp + 1
        Rhoe_tmp[ip] = Rhoe[ipp,1]
        Rhoe_tmp[ip+1] = Rhoe[ipp,2]
    end

    # ....
    # The rest of the code are similar to non-spin polarized case, however now
    # we use `Nspin=2` and pass `Rhoe_tmp` instead of `Rhoe`
end
\end{juliacode}

For PBE:
\begin{juliacode}
function calc_epsxc_PBE( pw::PWGrid, Rhoe::Array{Float64,1} )
    Npoints = size(Rhoe)[1]
    Nspin = 1

    # calculate gRhoe2
    gRhoe = op_nabla( pw, Rhoe )
    gRhoe2 = zeros( Float64, Npoints )
    for ip = 1:Npoints
        gRhoe2[ip] = dot( gRhoe[:,ip], gRhoe[:,ip] )
    end

    eps_x = zeros(Float64,Npoints)
    eps_c = zeros(Float64,Npoints)

    ptr = Libxc.xc_func_alloc()

    # exchange part
    Libxc.xc_func_init(ptr, Libxc.GGA_X_PBE, Nspin)
    Libxc.xc_gga_exc!(ptr, Npoints, Rhoe, gRhoe2, eps_x)
    Libxc.xc_func_end(ptr)

    # correlation part
    Libxc.xc_func_init(ptr, Libxc.GGA_C_PBE, Nspin)
    Libxc.xc_gga_exc!(ptr, Npoints, Rhoe, gRhoe2, eps_c)
    Libxc.xc_func_end(ptr)

    Libxc.xc_func_free(ptr)

    return eps_x + eps_c
end
\end{juliacode}

For PBE spin-polarized case

\begin{juliacode}
function calc_epsxc_PBE( pw::PWGrid, Rhoe::Array{Float64,2} )
    Nspin = size(Rhoe)[2]
    if Nspin == 1
        return calc_epsxc_PBE( pw, Rhoe[:,1] )
    end

    Npoints = size(Rhoe)[1]

    # calculate gRhoe2
    gRhoe_up = op_nabla( pw, Rhoe[:,1] )
    gRhoe_dn = op_nabla( pw, Rhoe[:,2] )
    gRhoe2 = zeros( Float64, 3*Npoints )
    ipp = 0
    for ip = 1:3:3*Npoints
        ipp = ipp + 1
        gRhoe2[ip]   = dot( gRhoe_up[:,ipp], gRhoe_up[:,ipp] )
        gRhoe2[ip+1] = dot( gRhoe_up[:,ipp], gRhoe_dn[:,ipp] )
        gRhoe2[ip+2] = dot( gRhoe_dn[:,ipp], gRhoe_dn[:,ipp] )
    end

    Rhoe_tmp = zeros(2*Npoints)
    ipp = 0
    for ip = 1:2:2*Npoints
        ipp = ipp + 1
        Rhoe_tmp[ip] = Rhoe[ipp,1]
        Rhoe_tmp[ip+1] = Rhoe[ipp,2]
    end

    # ....
    # The rest of the code are similar to non-spin polarized case, however now
    # we use `Nspin=2` and pass `Rhoe_tmp` instead of `Rhoe`
end
\end{juliacode}


\subsubsection{Calculation of $V_{\mathrm{xc}}$ in \textsf{PWDFT.jl} using Libxc}

VWN non-spin polarized:
\begin{juliacode}
function calc_Vxc_VWN( Rhoe::Array{Float64,1} )
    Npoints = size(Rhoe)[1]
    Nspin = 1
    v_x = zeros(Float64,Npoints)
    v_c = zeros(Float64,Npoints)

    ptr = Libxc.xc_func_alloc()

    # exchange part
    Libxc.xc_func_init(ptr, Libxc.LDA_X, Nspin)
    Libxc.xc_lda_vxc!(ptr, Npoints, Rhoe, v_x)
    Libxc.xc_func_end(ptr)

    # correlation part
    Libxc.xc_func_init(ptr, Libxc.LDA_C_VWN, Nspin)
    Libxc.xc_lda_vxc!(ptr, Npoints, Rhoe, v_c)
    Libxc.xc_func_end(ptr)

    Libxc.xc_func_free(ptr)

    return v_x + v_c
end
\end{juliacode}

VWN spin-polarized:
\begin{juliacode}
function calc_Vxc_VWN( Rhoe::Array{Float64,2} )
    Nspin = size(Rhoe)[2]
    if Nspin == 1
        return calc_Vxc_VWN( Rhoe[:,1] )
    end

    Npoints = size(Rhoe)[1]

    Vxc = zeros( Float64, Npoints, 2 )
    V_x = zeros( Float64, 2*Npoints )
    V_c = zeros( Float64, 2*Npoints )

    # This is the transposed version of Rhoe, use copy
    Rhoe_tmp = zeros(2*Npoints)
    ipp = 0
    for ip = 1:2:2*Npoints
        ipp = ipp + 1
        Rhoe_tmp[ip] = Rhoe[ipp,1]
        Rhoe_tmp[ip+1] = Rhoe[ipp,2]
    end

    ptr = Libxc.xc_func_alloc()

    # exchange part
    Libxc.xc_func_init(ptr, Libxc.LDA_X, Nspin)
    Libxc.xc_lda_vxc!(ptr, Npoints, Rhoe_tmp, V_x)
    Libxc.xc_func_end(ptr)

    # correlation part
    Libxc.xc_func_init(ptr, Libxc.LDA_C_VWN, Nspin)
    Libxc.xc_lda_vxc!(ptr, Npoints, Rhoe_tmp, V_c)
    Libxc.xc_func_end(ptr)

    Libxc.xc_func_free(ptr)

    ipp = 0
    for ip = 1:2:2*Npoints
        ipp = ipp + 1
        Vxc[ipp,1] = V_x[ip] + V_c[ip]
        Vxc[ipp,2] = V_x[ip+1] + V_c[ip+1]
    end
    return Vxc
end
\end{juliacode}

PBE non-spin-polarized:
\begin{juliacode}
function calc_Vxc_PBE( pw::PWGrid, Rhoe::Array{Float64,1} )
    Npoints = size(Rhoe)[1]
    Nspin = 1

    # calculate gRhoe2
    gRhoe = op_nabla( pw, Rhoe )
    gRhoe2 = zeros( Float64, Npoints )
    for ip = 1:Npoints
        gRhoe2[ip] = dot( gRhoe[:,ip], gRhoe[:,ip] )
    end

    V_x = zeros(Float64,Npoints)
    V_c = zeros(Float64,Npoints)
    V_xc = zeros(Float64,Npoints)

    Vg_x = zeros(Float64,Npoints)
    Vg_c = zeros(Float64,Npoints)
    Vg_xc = zeros(Float64,Npoints)

    ptr = Libxc.xc_func_alloc()

    # exchange part
    Libxc.xc_func_init(ptr, Libxc.GGA_X_PBE, Nspin)
    Libxc.xc_gga_vxc!(ptr, Npoints, Rhoe, gRhoe2, V_x, Vg_x)
    Libxc.xc_func_end(ptr)

    # correlation part
    Libxc.xc_func_init(ptr, Libxc.GGA_C_PBE, Nspin)
    Libxc.xc_gga_vxc!(ptr, Npoints, Rhoe, gRhoe2, V_c, Vg_c)
    Libxc.xc_func_end(ptr)

    V_xc = V_x + V_c
    Vg_xc = Vg_x + Vg_c

    # gradient correction
    h = zeros(Float64,3,Npoints)
    divh = zeros(Float64,Npoints)
    for ip = 1:Npoints
        h[1,ip] = Vg_xc[ip] * gRhoe[1,ip]
        h[2,ip] = Vg_xc[ip] * gRhoe[2,ip]
        h[3,ip] = Vg_xc[ip] * gRhoe[3,ip]
    end
    # div ( vgrho * gRhoe )
    divh = op_nabla_dot( pw, h )
    for ip = 1:Npoints
        V_xc[ip] = V_xc[ip] - 2.0*divh[ip]
    end

    return V_xc
end
\end{juliacode}

PBE spin-polarized:
\begin{juliacode}
function calc_Vxc_PBE( pw::PWGrid, Rhoe::Array{Float64,2} )
    Nspin = size(Rhoe)[2]
    if Nspin == 1
        return calc_Vxc_PBE( pw, Rhoe[:,1] )
    end

    Npoints = size(Rhoe)[1]

    # calculate gRhoe2
    gRhoe_up = op_nabla( pw, Rhoe[:,1] ) # gRhoe = ∇⋅Rhoe
    gRhoe_dn = op_nabla( pw, Rhoe[:,2] )
    gRhoe2 = zeros( Float64, 3*Npoints )
    ipp = 0
    for ip = 1:3:3*Npoints
        ipp = ipp + 1
        gRhoe2[ip]   = dot( gRhoe_up[:,ipp], gRhoe_up[:,ipp] )
        gRhoe2[ip+1] = dot( gRhoe_up[:,ipp], gRhoe_dn[:,ipp] )
        gRhoe2[ip+2] = dot( gRhoe_dn[:,ipp], gRhoe_dn[:,ipp] )
    end

    V_xc = zeros(Float64, Npoints, 2)
    V_x  = zeros(Float64, Npoints*2)
    V_c  = zeros(Float64, Npoints*2)

    Vg_xc = zeros(Float64, 3, Npoints)
    Vg_x  = zeros(Float64, 3*Npoints)
    Vg_c  = zeros(Float64, 3*Npoints)

    Rhoe_tmp = zeros(2*Npoints)
    ipp = 0
    for ip = 1:2:2*Npoints
        ipp = ipp + 1
        Rhoe_tmp[ip] = Rhoe[ipp,1]
        Rhoe_tmp[ip+1] = Rhoe[ipp,2]
    end

    ptr = Libxc.xc_func_alloc()

    # exchange part
    Libxc.xc_func_init(ptr, Libxc.GGA_X_PBE, Nspin)
    Libxc.xc_gga_vxc!(ptr, Npoints, Rhoe_tmp, gRhoe2, V_x, Vg_x)
    Libxc.xc_func_end(ptr)

    # correlation part
    Libxc.xc_func_init(ptr, Libxc.GGA_C_PBE, Nspin)
    Libxc.xc_gga_vxc!(ptr, Npoints, Rhoe_tmp, gRhoe2, V_c, Vg_c)
    Libxc.xc_func_end(ptr)

    ipp = 0
    for ip = 1:2:2*Npoints
        ipp = ipp + 1
        V_xc[ipp,1] = V_x[ip] + V_c[ip]
        V_xc[ipp,2] = V_x[ip+1] + V_c[ip+1]
    end

    Vg_xc = reshape(Vg_x + Vg_c, (3,Npoints))

    h = zeros(Float64,3,Npoints)
    divh = zeros(Float64,Npoints)

    # spin up
    for ip = 1:Npoints
        h[1,ip] = 2*Vg_xc[1,ip]*gRhoe_up[1,ip] + Vg_xc[2,ip]*gRhoe_dn[1,ip]
        h[2,ip] = 2*Vg_xc[1,ip]*gRhoe_up[2,ip] + Vg_xc[2,ip]*gRhoe_dn[2,ip]
        h[3,ip] = 2*Vg_xc[1,ip]*gRhoe_up[3,ip] + Vg_xc[2,ip]*gRhoe_dn[3,ip]
    end
    divh = op_nabla_dot( pw, h )
    for ip = 1:Npoints
        V_xc[ip,1] = V_xc[ip,1] - divh[ip]
    end

    # Spin down
    for ip = 1:Npoints
        h[1,ip] = 2*Vg_xc[3,ip]*gRhoe_dn[1,ip] + Vg_xc[2,ip]*gRhoe_up[1,ip]
        h[2,ip] = 2*Vg_xc[3,ip]*gRhoe_dn[2,ip] + Vg_xc[2,ip]*gRhoe_up[2,ip]
        h[3,ip] = 2*Vg_xc[3,ip]*gRhoe_dn[3,ip] + Vg_xc[2,ip]*gRhoe_up[3,ip]
    end
    divh = op_nabla_dot( pw, h )
    for ip = 1:Npoints
        V_xc[ip,2] = V_xc[ip,2] - divh[ip]
    end

    return V_xc
end
\end{juliacode}
