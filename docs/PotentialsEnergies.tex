\section{Potentials and energies}

In KSDFT approach \cite{Hohenberg1964,Kohn1965}, total energy per unit cell system
$E^{\mathrm{KS}}_{\mathrm{total}}$ can be written as
\begin{equation}
E^{\mathrm{KS}}_{\mathrm{total}} =
E_{\mathrm{kin}} + E_{\mathrm{ele-nuc}} +
E_{\mathrm{Ha}} + E_{\mathrm{xc}} + E_{\mathrm{nuc-nuc}}
\label{eq:E_KS_total}
\end{equation}

Kohn-Sham equations:
\begin{equation}
H_{\mathrm{KS}} \psi_{i\mathbf{k}}(\mathbf{r}) =
\epsilon_{i\mathbf{k}} \psi_{i\mathbf{k}}(\mathbf{r})
\end{equation}

\begin{figure}[H]
\centering
\begin{tikzpicture}
\node[mybox] (box) {%
\begin{minipage}{0.6\textwidth}%
\begin{minted}{julia}
mutable struct Potentials
    Ps_loc::Array{Float64,1}
    Hartree::Array{Float64,1}
    XC::Array{Float64,2}
    Total::Array{Float64,2}
end
\end{minted}
\end{minipage}
};
\node[fancytitle, right=10pt] at (box.north west) {\jlinline{Potentials} struct definition};
\end{tikzpicture}
\caption{Definition of \jlinline{Potentials}.}
\end{figure}


\begin{figure}[H]
\centering
\begin{tikzpicture}
\node[mybox] (box) {%
\begin{minipage}{0.6\textwidth}%
\begin{minted}{julia}
mutable struct Energies
    Kinetic::Float64
    Ps_loc::Float64
    Ps_nloc::Float64
    Hartree::Float64
    XC::Float64
    NN::Float64
    PspCore::Float64
    mTS::Float64
end
\end{minted}
\end{minipage}
};
\node[fancytitle, right=10pt] at (box.north west) {\jlinline{Energies} struct definition};
\end{tikzpicture}
\caption{Definition of \jlinline{Energies}.}
\end{figure}


Calculation of electron density and total energy

\subsection{XC energy and potential}

\textsf{PWDFT.jl} uses \textsf{Libxc.jl}\cite{Libxc.jl}, a Julia wrapper to
\textsf{Libxc}\cite{Marques2012,Lehtola2018}, to calculate exchange correlation
energy and potentials.

For LDA we have:
\begin{align}
E_{\mathrm{xc}}\left[\rho_{\sigma}\right] & = \int \epsilon^{\mathrm{HEG}}_{\mathrm{xc}}
\left[ \rho_{\sigma}(\mathbf{r}) \right]
\rho_{\mathrm{tot}}(\mathbf{r})\, \mathrm{d}\mathbf{r} \\
& = \int \left\{
\epsilon^{\mathrm{HEG}}_{\mathrm{x}} \left[ \rho_{\sigma}(\mathbf{r}) \right] +
\epsilon^{\mathrm{HEG}}_{\mathrm{c}} \left[ \rho_{\sigma}(\mathbf{r}) \right]
\right\}
\rho(\mathbf{r})\, \mathrm{d}\mathbf{r}
\end{align}

\begin{equation}
\delta E_{\mathrm{xc}}\left[\rho_{\sigma}\right] =
\sum_{\sigma} \int
\left(
\epsilon^{\mathrm{HEG}}_{\mathrm{xc}} +
\rho_{\mathrm{tot}} \frac{\partial}{\partial \rho_{\sigma}} \epsilon^{\mathrm{HEG}}_{\mathrm{xc}}
\right)
\, \mathrm{d}\mathbf{r}\,\delta \rho_{\sigma}
\end{equation}


Implementation in \textsf{PWDFT.jl} using Libxc

\begin{juliacode}
function calc_epsxc_VWN( Rhoe::Array{Float64,1} )
    Npoints = size(Rhoe)[1]
    Nspin = 1
    eps_x = zeros(Float64,Npoints)
    eps_c = zeros(Float64,Npoints)

    ptr = Libxc.xc_func_alloc()

    # exchange part
    Libxc.xc_func_init(ptr, Libxc.LDA_X, Nspin)
    Libxc.xc_lda_exc!(ptr, Npoints, Rhoe, eps_x)
    Libxc.xc_func_end(ptr)

    # correlation part
    Libxc.xc_func_init(ptr, Libxc.LDA_C_VWN, Nspin)
    Libxc.xc_lda_exc!(ptr, Npoints, Rhoe, eps_c)
    Libxc.xc_func_end(ptr)

    Libxc.xc_func_free(ptr)

    return eps_x + eps_c
end
\end{juliacode}


\begin{juliacode}
    function calc_epsxc_VWN( Rhoe::Array{Float64,2} )

        Nspin = size(Rhoe)[2]
        Npoints = size(Rhoe)[1]

        if Nspin == 1
            return calc_epsxc_VWN( Rhoe[:,1] )
        end

        # Do the transpose manually
        Rhoe_tmp = zeros(2*Npoints)
        ipp = 0
        for ip = 1:2:2*Npoints
            ipp = ipp + 1
            Rhoe_tmp[ip] = Rhoe[ipp,1]
            Rhoe_tmp[ip+1] = Rhoe[ipp,2]
        end


        eps_x = zeros(Float64,Npoints)
        eps_c = zeros(Float64,Npoints)

        ptr = Libxc.xc_func_alloc()
        # exchange part
        Libxc.xc_func_init(ptr, Libxc.LDA_X, Nspin)
        Libxc.xc_lda_exc!(ptr, Npoints, Rhoe_tmp, eps_x)
        Libxc.xc_func_end(ptr)

        #
        # correlation part
        Libxc.xc_func_init(ptr, Libxc.LDA_C_VWN, Nspin)
        Libxc.xc_lda_exc!(ptr, Npoints, Rhoe_tmp, eps_c)
        Libxc.xc_func_end(ptr)

        #
        Libxc.xc_func_free(ptr)

        return eps_x + eps_c
    end
\end{juliacode}
