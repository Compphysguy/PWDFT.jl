All atomic systems are assumed to be periodic.

The definition of type \texttt{Atoms} is given below.

\begin{Shaded}
\begin{Highlighting}[]
\NormalTok{mutable struct Atoms}
\NormalTok{    Natoms::}\DataTypeTok{Int64}
\NormalTok{    Nspecies::}\DataTypeTok{Int64}
\NormalTok{    positions::}\DataTypeTok{Array}\NormalTok{\{}\DataTypeTok{Float64}\NormalTok{,}\FloatTok{2}\NormalTok{\}}
\NormalTok{    atm2species::}\DataTypeTok{Array}\NormalTok{\{}\DataTypeTok{Int64}\NormalTok{,}\FloatTok{1}\NormalTok{\}}
\NormalTok{    atsymbs::}\DataTypeTok{Array}\NormalTok{\{}\DataTypeTok{String}\NormalTok{,}\FloatTok{1}\NormalTok{\}}
\NormalTok{    SpeciesSymbols::}\DataTypeTok{Array}\NormalTok{\{}\DataTypeTok{String}\NormalTok{,}\FloatTok{1}\NormalTok{\}}
\NormalTok{    LatVecs::}\DataTypeTok{Array}\NormalTok{\{}\DataTypeTok{Float64}\NormalTok{,}\FloatTok{2}\NormalTok{\}}
\NormalTok{    Zvals::}\DataTypeTok{Array}\NormalTok{\{}\DataTypeTok{Float64}\NormalTok{,}\FloatTok{1}\NormalTok{\}}
\KeywordTok{end}
\end{Highlighting}
\end{Shaded}

Information about \texttt{LatVecs} and \texttt{Zvals} are also available
from \texttt{PWGrid} and \texttt{PsPots}. They are included to reduce
number of required arguments to several functions.

Currently, the following functions are provided to initialize an
\texttt{Atoms}:

\begin{Shaded}
\begin{Highlighting}[]
\NormalTok{atoms = Atoms() }\CommentTok{# dummy constructor}
\NormalTok{atoms = init_atoms_xyz(filexyz; in_bohr=false, verbose=false)}
\NormalTok{atoms = init_atoms_xyz_string(filexyz; in_bohr=false, verbose=false)}
\end{Highlighting}
\end{Shaded}

Note that, \texttt{LatVecs} must be set manually by:

\begin{Shaded}
\begin{Highlighting}[]
\NormalTok{atoms.LatVecs = }\FloatTok{16}\NormalTok{*eye(}\FloatTok{3}\NormalTok{) }\CommentTok{# for example}
\end{Highlighting}
\end{Shaded}

\texttt{Zvals} is set when constructing \texttt{PWHamiltonian}.

Equation \begin{equation}
\frac{\alpha}{\beta}
\end{equation}
