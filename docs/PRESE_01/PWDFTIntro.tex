\documentclass[english,9pt]{beamer}

\setlength{\parskip}{\smallskipamount}
\setlength{\parindent}{0pt}

\setbeamersize{text margin left=5pt, text margin right=5pt}

\usepackage{amsmath}
\usepackage{amssymb}
\usepackage{braket}

\usepackage{minted}
\newminted{julia}{breaklines,fontsize=\footnotesize,texcomments=true}
\newminted{python}{breaklines,fontsize=\footnotesize,texcomments=true}
\newminted{bash}{breaklines,fontsize=\footnotesize,texcomments=true}
\newminted{text}{breaklines,fontsize=\footnotesize,texcomments=true}

\definecolor{mintedbg}{rgb}{0.95,0.95,0.95}
\usepackage{mdframed}

\BeforeBeginEnvironment{minted}{\begin{mdframed}[backgroundcolor=mintedbg]}
\AfterEndEnvironment{minted}{\end{mdframed}}

\setcounter{secnumdepth}{3}
\setcounter{tocdepth}{3}

\makeatletter

 \newcommand\makebeamertitle{\frame{\maketitle}}%
 % (ERT) argument for the TOC
 \AtBeginDocument{%
   \let\origtableofcontents=\tableofcontents
   \def\tableofcontents{\@ifnextchar[{\origtableofcontents}{\gobbletableofcontents}}
   \def\gobbletableofcontents#1{\origtableofcontents}
 }

\makeatother

\usepackage{babel}

\begin{document}


\title{\texttt{PWDFT.jl}: Density Functional Theory Calculations with Julia}
\author{Fadjar Fathurrahman}
\institute{
Engineering Physics Department \\
Research Center for Nanoscience and Nanotechnology \\
Institut Teknologi Bandung
}
\date{20 April 2018}

\frame{\titlepage}

\begin{frame}
\frametitle{Codes for DFT}

\begin{itemize}
\item using Fortran, C, C++: ABINIT, VASP, Quantum Espresso
\item Python: GPAW
\end{itemize}

\end{frame}


\begin{frame}
\frametitle{PWDFT.jl}

\begin{itemize}
\item A package for doing DFT calculations using Julia
\item plane wave basis + pseudopotentials (norm-conserving)
\item LDA and GGA functionals can be used (via LibXC)
\item symmetry detection (via SPGLIB)
\item algorithms: SCF and Emin (PCG and DCM)
\end{itemize}

\end{frame}


\begin{frame}
\frametitle{Aim}

\begin{itemize}
\item Friendly-to-develop DFT package
\item quick development for testing various algorithms
\end{itemize}

\end{frame}


\begin{frame}[fragile]
\frametitle{Example: hydrogen molecule in a box}

\begin{juliacode}
using PWDFT
atoms = Atoms( xyz_file="H2.xyz",
               LatVecs=gen_lattice_sc(16.0) )
pspfiles = ["../pseudopotentials/pade_gth/H-q1.gth"]
ecutwfc = 15.0
Ham = Hamiltonian( atoms, pspfiles, ecutwfc )
KS_solve_SCF!( Ham )
\end{juliacode}

\end{frame}


\begin{frame}[fragile]
\frametitle{Converged Kohn-Sham energy components}

\begin{columns}

\begin{column}{0.45\textwidth}
PWDFT.jl's result:
\begin{textcode}
-------------------------
Final Kohn-Sham energies:
-------------------------

Kinetic    energy:       1.0100082964
Ps_loc     energy:      -2.7127874818
Ps_nloc    energy:       0.0000000000
Hartree    energy:       0.9015209348
XC         energy:      -0.6314251867
PspCore    energy:      -0.0000012675
-------------------------------------
Electronic energy:      -1.4326847047
NN         energy:       0.3131700043
-------------------------------------
Total      energy:      -1.1195147004
\end{textcode}
\end{column}

\begin{column}{0.55\textwidth}

ABINIT's result:
\begin{textcode}
Components of total free energy (in Hartree) :

   Kinetic energy  =  1.01004059294567E+00
   Hartree energy  =  9.01545039301481E-01
   XC energy       = -6.31436384237843E-01
   Ewald energy    =  3.13170052325859E-01
   PspCore energy  = -1.26742500464741E-06
   Loc. psp. energy= -2.71283243086241E+00
   NL   psp  energy=  0.00000000000000E+00
   >>>>>>>>> Etotal= -1.11951439795224E+00
\end{textcode}
\end{column}

\end{columns}

\end{frame}


\begin{frame}
\frametitle{SCF options}

Various mixing methods

Various diagonalization schemes

ChebySCF

\end{frame}


\begin{frame}
\frametitle{Direct energy minimization}

For systems with band gaps:

Conjugate gradient

\end{frame}

\end{document}

