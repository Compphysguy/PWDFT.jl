\documentclass[english,9pt]{beamer}

\setlength{\parskip}{\smallskipamount}
\setlength{\parindent}{0pt}

\setbeamersize{text margin left=5pt, text margin right=5pt}

\usepackage{amsmath}
\usepackage{amssymb}
\usepackage{braket}

\usepackage{minted}
\newminted{julia}{breaklines,fontsize=\footnotesize,texcomments=true}
\newminted{python}{breaklines,fontsize=\footnotesize,texcomments=true}
\newminted{bash}{breaklines,fontsize=\footnotesize,texcomments=true}
\newminted{text}{breaklines,fontsize=\footnotesize,texcomments=true}

\definecolor{mintedbg}{rgb}{0.95,0.95,0.95}
\usepackage{mdframed}

\BeforeBeginEnvironment{minted}{\begin{mdframed}[backgroundcolor=mintedbg]}
\AfterEndEnvironment{minted}{\end{mdframed}}

\setcounter{secnumdepth}{3}
\setcounter{tocdepth}{3}

\makeatletter

 \newcommand\makebeamertitle{\frame{\maketitle}}%
 % (ERT) argument for the TOC
 \AtBeginDocument{%
   \let\origtableofcontents=\tableofcontents
   \def\tableofcontents{\@ifnextchar[{\origtableofcontents}{\gobbletableofcontents}}
   \def\gobbletableofcontents#1{\origtableofcontents}
 }

\makeatother

\usepackage{babel}

\begin{document}


\title{\texttt{PWDFT.jl}: Density Functional Theory Calculations with Julia}
\author{Fadjar Fathurrahman}
\institute{
Engineering Physics Department \\
Research Center for Nanoscience and Nanotechnology \\
Institut Teknologi Bandung
}
\date{20 April 2018}

\frame{\titlepage}

\begin{frame}
\frametitle{Codes for DFT}

\begin{itemize}
\item using Fortran, C, C++: ABINIT, VASP, Quantum Espresso
\item Python: GPAW
\end{itemize}

\end{frame}

\begin{frame}
\frametitle{Software packages for DFT calculations}

There are a lot of software packages for DFT calculations, for examples:

\begin{itemize}
\item Quantum ESPRESSO
\item VASP
\item ABINIT
\item Gaussian series: G03, G09, G16
\item NWchem
\end{itemize}

More extensive list:
\url{https://en.wikipedia.org/wiki/List_of_quantum_chemistry_and_solid-state_physics_software}


\end{frame}



\begin{frame}
\frametitle{Problems}

These packages are very helpful for doing various calculations based on DFT.

These packages are suitable for black-box-type calculations where we are only concerned about the results.

However they are generally quite difficult to be extended. For example if there are some new development
on the DFT functionals, users generally need to wait for the next release of the package to use them
(if these functionals are to be implemented at all).

Another example is when users want to develop custom post-processing tools. Users need to know in some detail
about how the data they are interested in is represented or implemented in the package's source code.

\end{frame}





\begin{frame}
\frametitle{PWDFT.jl}

\begin{itemize}
\item A package for doing DFT calculations using Julia
\item plane wave basis + pseudopotentials (norm-conserving)
\item LDA and GGA functionals can be used (via LibXC)
\item symmetry detection (via SPGLIB)
\item algorithms: SCF and Emin (PCG and DCM)
\end{itemize}

\end{frame}


\begin{frame}
\frametitle{Aims}

\begin{itemize}
\item Friendly-to-developers DFT package: enables quick implementation of various algorithms
\item educational purpose: simple yet powerful enough to carry out practical DFT calculations
for molecular and crystalline systems.
\end{itemize}

\end{frame}


\begin{frame}[fragile]
\frametitle{Example: hydrogen molecule in a box}

\begin{juliacode}
using PWDFT
atoms = Atoms( xyz_file="H2.xyz",
               LatVecs=gen_lattice_sc(16.0) )
pspfiles = ["../pseudopotentials/pade_gth/H-q1.gth"]
ecutwfc = 15.0
Ham = Hamiltonian( atoms, pspfiles, ecutwfc )
KS_solve_SCF!( Ham )
\end{juliacode}

\end{frame}


\begin{frame}[fragile]
\frametitle{Converged Kohn-Sham energy components}

\begin{columns}

\begin{column}{0.45\textwidth}
PWDFT.jl's result:
\begin{textcode}
-------------------------
Final Kohn-Sham energies:
-------------------------

Kinetic    energy:       1.0100082964
Ps_loc     energy:      -2.7127874818
Ps_nloc    energy:       0.0000000000
Hartree    energy:       0.9015209348
XC         energy:      -0.6314251867
PspCore    energy:      -0.0000012675
-------------------------------------
Electronic energy:      -1.4326847047
NN         energy:       0.3131700043
-------------------------------------
Total      energy:      -1.1195147004
\end{textcode}
\end{column}

\begin{column}{0.55\textwidth}

ABINIT's result:
\begin{textcode}
Components of total free energy (in Hartree) :

   Kinetic energy  =  1.01004059294567E+00
   Hartree energy  =  9.01545039301481E-01
   XC energy       = -6.31436384237843E-01
   Ewald energy    =  3.13170052325859E-01
   PspCore energy  = -1.26742500464741E-06
   Loc. psp. energy= -2.71283243086241E+00
   NL   psp  energy=  0.00000000000000E+00
   >>>>>>>>> Etotal= -1.11951439795224E+00
\end{textcode}
\end{column}

\end{columns}

\end{frame}


\begin{frame}
\frametitle{SCF options}

Various mixing methods

Various diagonalization schemes

ChebySCF

\end{frame}


\begin{frame}
\frametitle{Direct energy minimization}

For systems with band gaps:

Conjugate gradient

\end{frame}

\end{document}

