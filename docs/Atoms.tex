\section{Describing an atomic system}

\begin{markdown}
Because plane wave basis set is used, all atomic systems are assumed
to be periodic. For isolated molecular systems, a periodic bounding box
must be specified.

Currently, the definition of type `Atoms` is given below.

```julia
mutable struct Atoms
    Natoms::Int64
    Nspecies::Int64
    positions::Array{Float64,2}
    atm2species::Array{Int64,1}
    atsymbs::Array{String,1}
    SpeciesSymbols::Array{String,1}
    LatVecs::Array{Float64,2}
    Zvals::Array{Float64,1}
end
```

Information about `LatVecs` and `Zvals` are also available
from `PWGrid` and `PsPotGTH`, respectively.
They are included to reduce number of required arguments to several functions.

`LatVecs` represents lattice vectors $a_{1}$, $a_{2}$, dan $a_{3}$, stacked by column
in $3\times3$ matrix.

`Zvals` is set when constructing `PWHamiltonian`. The default value is `zeros(Nspecies)`.

Currently, the following functions are provided to initialize an `Atoms`:

- Using dummy constructor:
  ```julia
  atoms = Atoms()
  ```
  It is important to set other fields of `atoms` accordingly.

- Using xyz-like structure
  ```julia
  atoms = init_atoms_xyz(filexyz; in_bohr=false, verbose=false)
  atoms = init_atoms_xyz_string(filexyz; in_bohr=false, verbose=false)
  ```
  In the first function, `filexyz` is a string representing path of the xyz file while
  in the second function `filexyz` represent directly the content of xyz file.
  Example:
  ```julia
  # Initialize using an existing xyz file
  atoms = init_atoms_xyz("H2O.xyz")
  # Initialize using 'inline' xyz file
  atoms = init_atoms_xyz(
  """
  2

  O   -0.8  0.0  0.0
  O    0.8  0.0  0.0
  """
  ```

Note that, for both way `LatVecs` must be set manually by:
```julia
atoms.LatVecs = 16*eye(3) # for example
```
Currently there is no warning or check being performed to make sure that `LatVecs`
is defined properly. The default value is `zeros(3,3)`. So an error will happen
if an instance of `PWGrid` is constucted because we will try to invert a zero matrix.


Equation
\begin{equation}
\frac{\alpha}{\beta}
\end{equation}


\end{markdown}
